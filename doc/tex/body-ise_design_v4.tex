% =============================================================================

\ISE{4} requires $\RVXLEN = 64$
and adopts a {\em double} column-packed 
representation of state and round key matrices,
i.e., {\em two} columns (or $8$ elements) are packed into a $64$-bit word.
It is similar in principle to the SPARC~\cite[Page 109]{SPARC:16} ISE,
but improves on it by adhering to the
$2$ source and $1$ destination register format.
By sourcing two $64$-bit registers, and writing a single $64$-bit register,
a single instruction can 
accept  all  of the current round state as  input
and
produce half of the next    round state as output.

SPARC~\cite[Page 109]{SPARC:16} adds $ 9$
instructions ($4$ for encryption, $4$ for decryption, and $1$ auxiliary).
For example, \VERB{AES_EROUND01} and \VERB{AES_EROUND23}
produce columns $0$ and $1$ and columns $2$ and $3$
respectively.
Each instruction sources $3$ $64$-bit registers, and writes a single
$64$-bit register.
As shown in \REFFIG{fig:v4:pseudo}, \ISE{4} improves this by 
adding only $ 7$
instructions ($2$ for encryption, $2$ for decryption, and $3$ auxiliary).
This is realised by utilising the Equivalent Inverse Cipher representation
detailed in \cite[Section 5.3.5]{FIPS:197}.
This enables all of the round transformations to be applied in the same
order for both encryption and decryption.
The \AESFUNC{AddRoundKey} step can then lifted out of the
round function instructions (where otherwise it would appear in the middle of
the decryption round), and implemented using a base ISA \VERB{xor}
instruction.
The round key then no longer needs to be an input to the instruction,
meaning it only needs $2$ source register operands.
We then note that the nature of \AESFUNC{ShiftRows} means we do
not need separate instructions to compute the next values of
columns (0,1) or columns (2,3) as the SPARC instructions do.
Instead, we can simply reverse the order of the source register
operands, and get the same effect.
This is detailed in \REFFIG{fig:v4:pseudo}, and an example round
function is shown in \REFFIG{fig:v4:round}.

For example,
\VERB{saes.v4.encsm rd, rA, rB}
applies
\AESFUNC{SubBytes}, \AESFUNC{ShiftRow}, and \AESFUNC{MixColumn}  
to elements in a packed column and
produces the {\em next} round values for packed columns (0,1).
Executing
\VERB{saes.v4.encsm rd, rB, rA}, with no change in values of
\VERB{rA} or \VERB{rB}, will produce the next round state values for
packed columns (2, 3).

\REFFIG{fig:v4:round}
demonstrates that use of \ISE{4} to implement AES encryption requires
$ 6$ instructions per round:
$ 2$ \VERB{ld}           
     instructions to load the round key,
$ 2$ \VERB{xor}           
     instructions to apply \AESFUNC{AddRoundKey},
$ 2$ \VERB{saes.v4.encsm}  
     instructions to apply \AESFUNC{SubBytes}, \AESFUNC{ShiftRows}, and \AESFUNC{MixColumns}.
In the $Nr$-th round, which omits \AESFUNC{MixColumns},
     \VERB{saes.v4.encsm}
is replaced by 
     \VERB{saes.v4.encs}.
Note that use of the Equivalent Inverse Cipher representation
necessitates inclusion of the \VERB{saes.v4.imix} instruction, in order
to efficiently imply the inverse \AESFUNC{MixColumn} step to words
of the Key-Schedule.

% =============================================================================
