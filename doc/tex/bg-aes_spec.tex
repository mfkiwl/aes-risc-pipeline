% =============================================================================

% -----------------------------------------------------------------------------

\paragraph{Syntax.}

As a block cipher, AES defines two algorithms
\[
\begin{array}{lcl}
\ALG{Enc} &:& \SET{ 0, 1 }^{8 \cdot 4 \cdot Nk} \times \SET{ 0, 1 }^{8 \cdot 4 \cdot Nb} \rightarrow \SET{ 0, 1 }^{8 \cdot 4 \cdot Nb} \\
\ALG{Dec} &:& \SET{ 0, 1 }^{8 \cdot 4 \cdot Nk} \times \SET{ 0, 1 }^{8 \cdot 4 \cdot Nb} \rightarrow \SET{ 0, 1 }^{8 \cdot 4 \cdot Nb} \\
\end{array}
\]
such that
$
m = \ALG{Dec}( k, c = \ALG{Enc}( k, m ) ) .
$
That is, given a plaintext $m$ and cipher key $k$, \ALG{Enc} encrypts $m$ 
under $k$; given the same $k$, \ALG{Dec} will invert \ALG{Enc} and so the
{\em same} $m$ can be recovered from the associated ciphertext $c$.  
In addition, it defines an algorithm
\ALG{KeyExp}
that expands~\cite[Section 5.2]{FIPS:197} the cipher key into a sequence 
of round keys then used by
\ALG{Enc}
or
\ALG{Dec};
where appropriate, we use
\[
\begin{array}{lcl}
\ALG{Enc-KeyExp} &:& \SET{ 0, 1 }^{8 \cdot 4 \cdot Nk} \rightarrow \SET{ 0, 1 }^{( 8 \cdot 4 \cdot Nb ) \times ( Nr + 1 )} \\
\ALG{Dec-KeyExp} &:& \SET{ 0, 1 }^{8 \cdot 4 \cdot Nk} \rightarrow \SET{ 0, 1 }^{( 8 \cdot 4 \cdot Nb ) \times ( Nr + 1 )} \\
\end{array}
\]
to denote said algorithm as specialised to suit
\ALG{Enc}
and
\ALG{Dec}
respectively.

% -----------------------------------------------------------------------------

\paragraph{Parameterisation.}

An AES parameter set~\cite[Figure 4]{FIPS:197}
is a triple
$
\TUPLE{ Nk, Nb, Nr }
$
where 
$Nk$ dictates the number of $32$-bit words in $k$,
$Nb$ dictates the number of $32$-bit words in $m$ or $c$ (i.e., a block),
and
$Nr$ dictates the number of rounds.  The standard AES parameter sets are
\[
\begin{array}{lcl}
\mbox{AES-128} &\mapsto& \TUPLE{ 4, 4, 10 } \\
\mbox{AES-192} &\mapsto& \TUPLE{ 6, 4, 12 } \\
\mbox{AES-256} &\mapsto& \TUPLE{ 8, 4, 14 } \\
\end{array}
\]
such that the number of bits in a plaintext (resp. ciphertext) block is fixed to 
$
8 \cdot 4 \cdot Nb = 128 .
$
From here on, we focus wlog. on encryption using AES-128 (other parameter 
sets are catered for naturally, and decryption with minor differences) so
use the terms AES and AES-128 synonymously.

% -----------------------------------------------------------------------------

\paragraph{Design.}

The mathematics underpinning AES are described in ~\cite[Section 4]{FIPS:197}.
In particular, it can be defined in terms of 
operations in the finite field $\F_{2^{  8}}$ constructed as
$
\F_{2}[\IND{x}] / ( \IND{x}^{8} + \IND{x}^{4} + \IND{x}^{3} + \IND{x} + 1 ) .
$
A hexadecimal short-hand~\cite[Section 3.2]{FIPS:197} is used to represent 
field literals, e.g.,
$
\AESCONST{13} ~\mapsto~ \RADIX{13}{16} ~\equiv~ \RADIX{00010011}{2} ~\mapsto~ \IND{x}^4 + \IND{x} + 1 .
$
Field 
      addition, 
multiplication, 
and  
      division
are denoted by
$\AESADD$,
$\AESMUL$,
and
$\AESINV$
respectively,
with the multiplication-by-$\IND{x}$ operation~\cite[Section 4.2.1]{FIPS:197} 
denoted \AESFUNC{xtime}.
Elements of $\F_{2^8}$ are collected into $( 4 \times 4 )$-element state
and round key matrices; the $i$-th row and $j$-th column of such a matrix 
relating to round $r$ is denoted
$\AESRND {s}{r}_{i,j}$
and
$\AESRND{rk}{r}_{i,j}$
respectively, with super- and/or subscripts omitted whenever irrelevant.

AES is an iterative block cipher, based on a substitution-permutation network.
This means encryption using AES can be described~\cite[Section 5.2]{FIPS:197}
as follows:
1)    the  input  plaintext is pre-whitened to yield
      $\AESRND {s}{  0} = m \AESADD \AESRND{rk}{0} = m \AESADD k$,
2)    each $r$-th round, for $1 \leq r \leq Nr$, demands computation of
      $\AESRND {s}{r+1} = \ALG{P-layer}( \ALG{S-layer}( \AESRND{s}{r}                        ) ) \AESADD \AESRND{rk}{r}$,
      and therefore use of round key
      $\AESRND{rk}{r  }$,
3)    the output ciphertext is
      $c = \AESRND{s}{Nr}$.
Note that an alternative round definition, namely
      $\AESRND {s}{r+1} = \ALG{P-layer}( \ALG{S-layer}( \AESRND{s}{r} \AESADD \AESRND{rk}{r} ) )                       $ ,
is plausible: this shifts the 
 pre-whitening step {\em before} 2) 
into an analogous 
post-whitening step {\em  after} 2)
to yield an equivalent result.
At a  low(er) level,
the computation of each round is specified via four round functions (each of 
which has an inverse, to support decryption):

\begin{itemize}

\item \AESFUNC{SubBytes}
      ~\cite[Section 5.1.1]{FIPS:197}
      operates element-wise,
      computing
      $\AESRND{s}{r+1}_{i,j} = \ALG{S-box}( \AESRND{s}{r}_{i,j} )$
      via application of the S-box:
      given an element $x$, this component can be described as
      \[
      \begin{array}{lcl}
      \ALG{S-Box} &:& \left\{\begin{array}{ccc}
                             \F_{2^8} &\rightarrow& \F_{2^8} \\
                             x        &\mapsto    & f(g(x))  \\
                             \end{array}
                      \right.
      \end{array}
      \]
      where 
      $g$ is an inversion, 
      and 
      $f$ is a specially selected affine transformation.
      Where appropriate,
      we overload \AESFUNC{SubBytes} by allowing it to denote application 
      of the S-box to {\em any} collection, 
      e.g., a row, column, or, more generally, a sequence, 
      of elements.

\item \AESFUNC{ShiftRows}
      ~\cite[Section 5.1.2]{FIPS:197}
      operates     row-wise,
      rotating each 
      $i$-th row 
      of 
      $\AESRND{s}{r  }$
      by $i$ elements
      to form 
      the associated row    of
      $\AESRND{s}{r+1}$,
      i.e.,
      $\AESRND{s}{r+1}_{i,j} = \AESRND{s}{r}_{i,j + i \pmod{Nb}}$.
      Where appropriate,
      we use
      \AESFUNC{ShiftRow}
      to denote
      the operation applied to a single 
      row
      within \AESFUNC{ShiftRows}.

\item \AESFUNC{MixColumns}
      ~\cite[Section 5.1.3]{FIPS:197}
      operates  column-wise,
      multiplying each 
      $j$-th column
      of 
      $\AESRND{s}{r  }$
      with a constant MDS matrix
      to form 
      the associated column of
      $\AESRND{s}{r+1}$.
      Where appropriate,
      we use
      \AESFUNC{MixColumn}
      to denote
      the operation applied to a single 
      column 
      within \AESFUNC{MixColumns}, i.e., multiplication of a $4$-element 
      column vector by the constant MDS matrix.
      
\item \AESFUNC{AddRoundKey}
      ~\cite[Section 5.1.4]{FIPS:197}
      operates element-wise,
      computing
      $\AESRND{s}{r+1}_{i,j} = \AESRND{s}{r}_{i,j} \AESADD \AESRND{rk}{r}_{i,j}$ 
      and thereby mixing a round key into the state.

\end{itemize}

\noindent
Note that
$
\ALG{S-layer} = \AESFUNC{SubBytes} ,
$
and
\[
\ALG{P-layer} = \left\{\begin{array}{l@{\;}c@{\;}l lr}
                       \AESFUNC{MixColumns} &\circ& \AESFUNC{ShiftRows} & \mbox{in rounds} & 1 \leq r < Nr \\
                                            &     & \AESFUNC{ShiftRows} & \mbox{in round } &            Nr \\
                       \end{array}
                \right.
\]
i.e., the last, $Nr$-th round differs from the initial $Nr - 1$ rounds.  As
such, a round as defined above is constructed via
$
\AESFUNC{AddRoundKey} \circ \AESFUNC{MixColumns} \circ \AESFUNC{ShiftRows} \circ \AESFUNC{SubBytes} 
$
or
$
\AESFUNC{AddRoundKey} \circ                            \AESFUNC{ShiftRows} \circ \AESFUNC{SubBytes}
$
respectively, where, because \AESFUNC{ShiftRows} and \AESFUNC{SubBytes}
commute, the order they are applied in can be selected to suit.

