% =============================================================================

Here, we survey AES-related ISE designs split into
1) industry-specified ISEs,
   which are {\em     standard} extensions,
   and
2) academia-specified ISEs,
   which are {\em non-standard} extensions,
wrt. a given base ISA.
   Each ISE is classified as either
   workload-specific,
   if it is only useful for AES,
   or
   workload-agnostic,
   if it is      useful for AES and other workloads.
Note that we exclude work where an ISE for another workload can be applied 
{\em  to} AES
but was not designed 
{\em for} AES
(see, e.g., Tillich and Gro{\ss}sch\"{a}dl~\cite{TilGro:04} who apply an ISE intended for ECC to AES).

% =============================================================================

\subsubsection{Standard, industry-specified ISEs}

\noindent
{\bf Intel}
      introduced support for AES in 
      x86
      per~\cite[Section 12.13]{X86:1:18}.
      Instructions use a
          destructive $2$-address ($1$ source, $1$ source/destination)  
      or
      non-destructive $3$-address ($2$ source, $1$        destination)
      format
      depending on the variant (e.g., XMM- vs. AVX-based),
      and operate on data housed in the pre-existing
      vector 
      register file, implying $R = 128$.
      AES is implemented by
      1) adopting a 
          fully-packed
         representation of state and round key matrices,
         then
      2) using
             \VERB{AESENC}         ~\cite[Page 3-54]{X86:2:18}
         to construct a round implementation as
         \[
         \VERB{AESENC} \mapsto \AESFUNC{AddRoundKey} \circ \AESFUNC{MixColumns} \circ \AESFUNC{SubBytes} \circ \AESFUNC{ShiftRows}
         \]
%     Note that
%            \VERB{AESENCLAST}     ~\cite[Page 3-56]{X86:2:18}
%     supports 
%     the $Nr$-th round;
%     additional instructions are provided to 
%     support
%     decryption
%     (i.e., \VERB{AESDEC}         ~\cite[Page 3-50]{X86:2:18}
%            and
%            \VERB{AESDECLAST}     ~\cite[Page 3-52]{X86:2:18})
%     and
%     key expansion
%     (i.e., \VERB{AESKEYGENASSIST}~\cite[Page 3-59]{X86:2:18}
%            and
%            \VERB{AESIMC}         ~\cite[Page 3-58]{X86:2:18}).

\noindent
{\bf IBM}
      introduced support for AES in 
      POWER
      per~\cite[Section 6.11.1]{POWER:18}.
      Instructions use a
      non-destructive $3$-register ($2$ source, $1$        destination)
      format,
      and operate on data housed in the pre-existing
      vector 
      register file, implying $R = 128$.
      AES is implemented by
      1) adopting a 
          fully-packed
         representation of state and round key matrices,
         then
      2) using
             \VERB{vcipher}     ~\cite[Page 304]{POWER:18}
         to construct a round implementation as
         \[
         \VERB{vcipher} \mapsto \AESFUNC{AddRoundKey} \circ \AESFUNC{MixColumns} \circ \AESFUNC{ShiftRows} \circ \AESFUNC{SubBytes}
         \]
%     Note that
%            \VERB{vcipherlast} ~\cite[Page 304]{POWER:18}
%     supports 
%     the $Nr$-th round;
%     additional instructions are provided to 
%     support
%     decryption
%     (i.e., \VERB{vncipher}    ~\cite[Page 305]{POWER:18}
%            and
%            \VERB{vncipherlast}~\cite[Page 305]{POWER:18})
%     and
%     key expansion
%     (i.e., \VERB{vsbox}       ~\cite[Page 305]{POWER:18}).

\noindent
{\bf ARM}
      introduced support for AES in 
      ARMv8-A
      per~\cite[Section A2.3]{ARMv8-A:20}.
      Instructions use a
          destructive $2$-address ($1$ source, $1$ source/destination)  
      format,
      and operate on data housed in the pre-existing
      vector 
      register file, implying $R = 128$.
      AES is implemented by
      1) adopting a 
          fully-packed
         representation of state and round key matrices,
         then
      2) using
             \VERB{AESE}  ~\cite[Section C7.2.8 ]{ARMv8-A:20}
             and
             \VERB{AESMC} ~\cite[Section C7.2.10]{ARMv8-A:20}
         to construct a round implementation as
         \[
         \VERB{AESMC} \circ \VERB{AESE} \mapsto \AESFUNC{MixColumns} \circ ( \AESFUNC{SubBytes} \circ \AESFUNC{ShiftRows} \circ \AESFUNC{AddRoundKey} ) ,
         \]
%         where the alternative round definition from 
%         \REFSEC{sec:bg:aes_spec} 
%         is assumed to cater for the order of application.
%     Note that
%     additional instructions are provided to 
%     support
%     decryption
%     (i.e., \VERB{AESD}  ~\cite[Section C7.2.7 ]{ARMv8-A:20}
%            and
%            \VERB{AESIMC}~\cite[Section C7.2.9 ]{ARMv8-A:20}),
%     but none are required to 
%     support
%     the $Nr$-th round:
%     \VERB{AESE} obviously lacks \AESFUNC{MixColumns}, and the post-whitening 
%     step is naturally supported via XOR. 

\noindent
{\bf Oracle}
      introduced support for AES in 
      SPARC 
      per~\cite[Sections 7.3+7.4]{SPARC:16}.
      Instructions use a
      non-destructive $4$-address ($3$ source, $1$        destination)
      format,
      and operate on data housed in the pre-existing
      general-purpose
      register file, implying $R =  64$.
      AES is implemented by
      1) using a 
         column-packed
         representation of state and round key matrices,
         then
      2) using
             \VERB{AES_EROUND01}     ~\cite[Page 109]{SPARC:16}
             and
             \VERB{AES_EROUND23}     ~\cite[Page 109]{SPARC:16}
         to construct a round implementation as
         \[
         ( \VERB{AES_EROUND01};\VERB{AES_EROUND23} ) \mapsto \AESFUNC{AddRoundKey} \circ \AESFUNC{MixColumns} \circ \AESFUNC{ShiftRows} \circ \AESFUNC{SubBytes} 
         \]
         in two steps:
         the first  step processes columns $0$ and $1$ via \VERB{AES_EROUND01}
         whereas
         the second step processes columns $2$ and $3$ via \VERB{AES_EROUND23}.
%     Note that
%            \VERB{AES_EROUND01_LAST}~\cite[Page 109]{SPARC:16}
%            and
%            \VERB{AES_EROUND23_LAST}~\cite[Page 109]{SPARC:16}
%     support 
%     the $Nr$-th round;
%     additional instructions are provided to 
%     support
%     decryption
%     (i.e., \VERB{AES_DROUND01}     ~\cite[Page 109]{SPARC:16},
%            \VERB{AES_DROUND23}     ~\cite[Page 109]{SPARC:16},
%            \VERB{AES_DROUND01_LAST}~\cite[Page 109]{SPARC:16},
%            and
%            \VERB{AES_DROUND23_LAST}~\cite[Page 109]{SPARC:16})
%     and
%     key expansion
%     (i.e., \VERB{AES_KEXPAND0}     ~\cite[Page 112]{SPARC:16},
%            \VERB{AES_KEXPAND1}     ~\cite[Page 109]{SPARC:16},
%            and
%            \VERB{AES_KEXPAND2}     ~\cite[Page 112]{SPARC:16}).

% -----------------------------------------------------------------------------

\subsubsection{Non-standard, academia-specified ISEs}

% workload-agnostic

      Burke et al.~\cite{BurMcDAus:00}
      propose 
      a workload-agnostic ISE
      based on workload characterisation for the
      DEC Alpha architecture \cite{alpha2014alpha}.
      Per~\cite{BurMcDAus:00}, pertinent examples
      for AES
      include
      a) \VERB{ROL}
         and
         \VERB{ROR},
         which perform
         left- and right-rotate,
         and
      b) \VERB{SBOX},
         which 
         extracts elements to form look-up table offsets.
         In one configuration,
         the resulting memory accesses are supported by a
         set of special-purpose ``S-box caches''.

      Fiskiran and Lee~\cite{FisLee:05}
      propose 
      a workload-agnostic ISE
      that employs a so-called
      Parallel Table Lookup Module (PTLU) for a ``{\em RISC like}''
      instruction set.
      For AES, 
      this accelerates implementations based on T-tables 
      by affording an addressing mode that
      a) integrates 
         extraction of elements to form look-up table offsets,
         and
      b) performs the associated table look-ups in parallel, supported by
         a dedicated scratch-pad memory.

      Biham et al.~\cite[Page 232]{BihAndKnu:98}
      propose (in theory)
      and
      Grabher et al.~\cite{GraGroPag:08}
      explore  (in practice)
      a workload-agnostic ISE
      that supports bit-sliced implementations for their custom
      CRISP (``{\em RISC like}'') architecture.
      The ISE allows computation using 
      {\em configurable} $4$-input, $2$-output 
      Boolean functions, vs. 
      {\em fixed}        $2$-input, $1$-output alternatives such as NOT, AND, OR, and XOR.
      Sequences of native Boolean instructions, which dominate bit-sliced
      implementations, can thereby be ``compressed'' into use of the ISE.
      Doing so improves both latency and footprint.
      \cite[Section 4]{GraGroPag:08} details the application to AES.

% workload-specific

      Nadehara et al.~\cite{NadIkeKur:04} 
      propose 
      a workload-specific ISE
       that could be described as 
      ``hardware-assisted T-tables'':
      observing that $\forall x, i \neq j$, $T_i[ x ]$ is a rotation of
      $T_j[ x ]$, they support on-the-fly computation (vs. via look-up)
      of T-table entries.
      The ISE constitutes a single instruction
      $\VERB{AESENC} \mapsto T_i$,
      supported by a dedicated hardware module
      (see~\cite[Figure 6]{NadIkeKur:04}).
      Instances of \VERB{AESENC}
      1) extract an   input element from a 
         packed  input column
      2) use the input to compute an output element equivalent to a
         look-up from the T-table,
         and
      3) store   the output element into a
         packed output column.
      This approach was reapplied by Saarinen~\cite{Saarinen:20}
      within the context of RISC-V.

      Tillich et al.~\cite{TilGroSze:05}
      propose 
      a workload-specific ISE
       that could be described as 
      ``hardware-assisted S-box'' for the SPARC V8 architecture.
      The ISE constitutes a single instruction
      $\VERB{sbox} \mapsto \AESFUNC{SubBytes}$,
      supported by a dedicated hardware module
      (see~\cite[Figure 1]{TilGroSze:05}).
      Instances of \VERB{sbox}
      1) extract an   input element from a packed  input row or column,
      2) use the input to compute an output element equivalent to a
         look-up from the S-box,
         and
      3)  insert the output element into a packed output row or column.
         Using insert vs. overwrite semantics allows
         \AESFUNC{ShiftRows} to be computed {\em for free}.

      Bertoni et al.~\cite{BBFR:06}
      propose 
      a workload-specific ISE
       that could be described as 
      ``hardware-assisted round functions''.
      The ISE includes
      1) zero-overhead rotation (similar to ARM),
         and
      2) byte- and word-oriented variants of
         $\VERB{SMix} \mapsto \AESFUNC{MixColumn} \circ \AESFUNC{SubBytes}$.
      
      Tillich and Gro{\ss}sch\"{a}dl~\cite{TilGro:06}
      propose 
      a workload-specific ISE
       that could be described as 
      ``hardware-assisted round functions'' for the SPARC V8 architecture.
      The ISE includes
         byte- and word-oriented variants of
         $\VERB  {sbox[4][s|r]} \mapsto \AESFUNC{SubBytes} $
         and
         $\VERB{mixcol[4][s]  } \mapsto \AESFUNC{MixColumn}$;
      per~\cite[Section 4.3]{TilGro:06},
      the most efficient variant allows
         a zero-overhead implementation of \AESFUNC{ShiftRows} to be realised.


% =============================================================================
