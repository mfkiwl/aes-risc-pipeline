% =============================================================================

%Although differing in nature, both AES and RISC-V represent important
%standards.  In this paper, we have addressed the challenge of secure, 
%efficient implementation of AES on RISC-V: our approach harnesses the
%modularity afforded by RISC-V, through a focus on the use of ISEs.

Motivated by ongoing efforts to standardise support 
for AES in RISC-V, we have implemented and evaluated five ISE designs 
on two different RISC-V compliant base micro-architectures.
Our conclusion is that
1) \ISE{3}
   is the best option for 
   AES on $32$-bit cores,
2) \ISE{4}
   is the best option for 
   AES on $64$-bit cores,
   and
3) the
   standard 
   B~\cite[Section 17]{RV:ISA:I:19}
   extension
   can combine with either option to support AES-GCM.

\paragraph{Future Work.}
Our evaluations of the different ISEs have focused primarily on
performance, code size and hardware cost metrics.
Because our work is a departure from historic
AES ISEs in that they are designed to be suitable for small, embedded
CPUs, power and EM side-channel security will likely be a consideration
for implementations of these ISEs.
We consider the problem of side-channel secure ISE design to be an
open problem, particularly in terms of making the same code portably
side-channel secure across multiple implementations of the same ISE.
Future efforts would be well spent in studying this problem, perhaps
looking at creating custom extensions based on the recommendations here
to support side-channel security.

% TODO: maybe try to add some insight into ISA design constraints ...
%
%The requirement for $3$-address (i.e., $2$ source and $1$ destination)
%instruction format prevented some further optimisations, 
%e.g., the integration of \AESFUNC{AddRoundKey} in \ISE{4}.

% =============================================================================
