% =============================================================================

Each ISE variant was integrated into the two host cores 
described in \REFSEC{sec:ise:imp}.
The variants which assume  $\RVXLEN = 32$
(\ISE{1}, \ISE{2}, \ISE{3}, and \ISE{5}) 
were evaluated
on {\em both} the
$32$-bit \CORE{2} core
{\em  and} the
$32$-bit \CORE{1} core;
the variant  which assumes $\RVXLEN = 64$
(\ISE{4})
was  evaluated
on {\em only} the
$64$-bit \CORE{1} core.
For \ISE{1}, \ISE{2} and \ISE{5} a trade-off
between latency and area exists. 
Each such case is considered through two optimisation goals:
the (A)rea    goal
instantiates $1$ S-box   and has a $n$-cycle execution latency,
whereas
the (L)atency goal
instantiates $4$ S-boxes and has a $1$-cycle execution latency.

\REFTAB{tab:eval:hw:encdec}
shows the metrics associated with the hardware implementations, 
and show the separated cost of the standalone ISE logic, and the
cost of the core and integrated ISE.
We use the open source Yosys~\cite{yosys} synthesis tool (v0.9+1706)
with default settings
to provide post-synthesis (as opposed to post-layout) circuit area in the
form of NAND2 gate equivalents and circuit depths in the form of gate delays.
While more abstract than providing exact area and
frequency results for a particular ASIC standard cell library, is
much easier to reproduce\footnote{
Especially
for researchers lacking expensive commercial
synthesis tools and process design kits.
} while still providing meaningful results.
We focus on ASIC implementations (rather than FPGA implementations)
because this is the more relevant metric to the industrial (rather than
academic) RISC-V community.
This methodology has also been used for other RISC-V standard extension
proposals, namely the bit-manipulation extension~\cite[Section 3.1, Page 54]{riscv:bitmanip:draft}.
We found 
that none of the ISEs affected the critical gate delay path of either the
\CORE{2} or \CORE{1} core.
Considering each ISE as implemented on the \CORE{1} core, we note the 
overhead wrt. area is marginal: this stems from the fact that the 
baseline area of \CORE{1} includes the data and instruction caches.

In
\REFTAB{tab:eval:hw:dec}
we consider the hardware costs when only {\em encryption} instructions are
implemented.
This is relevant to systems which only care about certain block cipher
modes of operation, such as Galos/Counter-mode,
which only use the encryption function of a block cipher.
We discuss this further in \REFSEC{sec:gcm}.

% =============================================================================
