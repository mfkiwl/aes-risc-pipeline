% =============================================================================

\REFTAB{tab:eval:hw:encdec}
demonstrates that all ISE variants
imply a modest area overhead relative to their host core.
For the RV32 \CORE{1} the area overhead of a synthesised \CORE{1} Tile with
caches was less than $1\%$ in all cases.
For the \CORE{2}, the area overhead ranged between
$13\%$ (\ISE{5} (L))
and
$3\%$ (\ISE{3}).
\REFTAB{tab:eval:sw:size}
shows all ISE variants
having similarly small memory footprints in terms of both instruction code and
data.
Beyond this, and per 
\REFSEC{sec:ise:design},
the primary metric of interest is efficiency in terms of
the latency-area product.
this metric draws on data from
\REFTAB{tab:eval:hw:encdec}
plus either
\REFTAB{tab:eval:sw:perf:2}
or
\REFTAB{tab:eval:sw:perf:1}
for the \CORE{2} or \CORE{1} core respectively.
We deliberately omit the area of the host core from this calculation, as this
fixed overhead dominates the final value and detracts from the comparison
between ISEs themselves.

\REFTAB{tab:eval:results} 
captures the results for the \CORE{1} core, although the same conclusion can 
be drawn for the \CORE{2} core.  Qualitatively, we place more of a weight on 
Encryption (\ALG{Enc})
and 
Decryption (\ALG{Dec})
vs.
Encryption Key Expansion (\ALG{Enc-KeyExp})
and 
Decryption Key Expansion (\ALG{Dec-KeyExp}),
because
typically many \ALG{Enc} or \ALG{Dec} operations are performed per
\ALG{KeyExp}.

For a $32$-bit core, our conclusion is that
\ISE{3} 
is the best option.
Despite not being the fastest (by a small margin), it is the most efficient,
and simplest to implement.
The area optimised \ISE{2} implementation sometimes comes close in
efficiency, but requires a more complex multi-cycle implementation
in this case.
We note that \ISE{3} has relatively poor performance for the decryption
key schedule.
This is because it uses the Equivalent Inverse Cipher representation,
and must first create an {\em encryption} orientated key schedule, before
applying the Inverse \AESFUNC{MixColumns} transform to each word in the key schedule.
Each word requires $8$ instructions to apply {\em only} the Inverse \AESFUNC{MixColumns} 
transform. We believe this is reasonable, as one typically performs
many block decryptions per key schedule operation.

Compared to past work, our implementation of \ISE{3} is
slightly smaller than its original description in \cite{Saarinen:20}:
$1149$ v.s. $1240$ gates.
\cite{Saarinen:20} estimates a $5\times$ performance improvement, which is
slightly better than our measured $4\times$ improvement, though this is
dependant on relative memory access latencies.
We would expect this improvement to increase in systems which store TTables
in (relatively) high latency flash memory.
\ISE{3} performs considerably better than \cite{TilGroSze:05},
which achieves only a $2\times$ speedup in the best case.

We note that despite needing the same number of instructions per round
as \ISE{2} (based on \cite{TilGroSze:05}), our \ISE{5} design suffers in terms
of performance.
This is due to the conversion between quadrant-packed and column packed
representations.

For a $64$-bit core,
\ISE{4} 
is the best option, which is somewhat obvious because it specifically makes
use of the wider data-path.
It is $10\times$ faster to perform a block encryption than a baseline T-Table
implementation targeting a 64-bit base RISC-V architecture.
With reference to
\REFTAB{tab:eval:sw:perf:1}, 
note that the number of cycles per instruction executed is relatively high.
This fact stems from use of the ROCC interface, in that forwarding of the 
result from an ISE instruction (that uses the ROCC) incurs an overhead vs. 
an ISE instruction; fine-grained integration of the AES-FU could therefore
incrementally improve the results.

We believe it is sensible to standardise different ISEs for the
RV32 and RV64 base ISAs.
This allows each ISE design to better suit the constraints of each
base ISA.
In the RV32 case, this acknowledges that such cores will most often
appear in resource-constrained, embedded or IoT class devices.
Hence, the most efficient ISE design is appropriate.
For necessarily larger RV64-based designs, it makes sense to take advantage
of the wider data-path, and acknowledge that these are more likely to
be application class cores. Hence, they will place a higher value
on performance than area-efficiency.

% -----------------------------------------------------------------------------

\begin{adjustbox}{center,caption={
    Hardware metrics for each ISE variant with
    encrypt and decrypt instructions.
                                 },label={tab:eval:hw:encdec},float={table}[!t]}
\centering
\begin{tabular}{|c|l|rr|r|r|}
\hline
  \multicolumn{1}{|c|}{ISA}
& \multicolumn{1}{ c|}{Variant}
& \multicolumn{1}{ c|}{             ISE}
& \multicolumn{1}{ c|}{       ISE Path }
& \multicolumn{1}{ c|}{\CORE{2}     CPU}
& \multicolumn{1}{ c|}{\CORE{1}     CPU}
\\
& \multicolumn{1}{ c|}{/ Goal       }
& \multicolumn{1}{ c|}{Area         }
& \multicolumn{1}{ c|}{Depth        }
& \multicolumn{1}{ c|}{$+$ ISE area }
& \multicolumn{1}{ c|}{$+$ ISE area }
\\
\hline
\hline
 RV32IMC & Baseline    &              &            &       37325  ($1.00\times$) &       3501576 ($1.000\times$) \\
 RV32IMC & \ISE{1} (L) &        3514  & \bftab 18  &       41746  ($1.12\times$) &       3508448 ($1.002\times$) \\
 RV32IMC & \ISE{1} (A) &        2195  &        21  &       40171  ($1.08\times$) &       3506995 ($1.002\times$) \\
 RV32IMC & \ISE{2} (L) &        3574  &        19  &       41132  ($1.10\times$) &       3508946 ($1.002\times$) \\
 RV32IMC & \ISE{2} (A) &        1355  &        21  &       38777  ($1.04\times$) &\bftab 3506591 ($1.001\times$) \\
 RV32IMC & \ISE{3}     & \bftab 1149  &        30  &\bftab 38546  ($1.03\times$) &       3506761 ($1.001\times$) \\
 RV32IMC & \ISE{5} (L) &        4172  &        21  &       42035  ($1.13\times$) &       3510055 ($1.002\times$) \\
 RV32IMC & \ISE{5} (A) &        1726  &        23  &       39144  ($1.05\times$) &       3507755 ($1.002\times$) \\
\hline
\hline
 RV64IMC & Baseline &          &          &  N/A  & 3717607 (1.000$\times$) \\
 RV64IMC & \ISE{4}  &     8226 &       28 &  N/A  & 3733786 (1.004$\times$) \\
\hline
\end{tabular}
\end{adjustbox}


\begin{adjustbox}{center,caption={Software  memory footprint measured in bytes
                                  for each ISE variant.
                                 },label={tab:eval:sw:size},float={table}[!t]}
\centering
\begin{tabular}{|c|c|r|r|r|r|r|}
\hline
  \multicolumn{1}{|c|}{ISA}
& \multicolumn{1}{ c|}{Variant}
& \multicolumn{1}{ c|}{$\ALG{Enc}$}
& \multicolumn{1}{ c|}{$\ALG{Dec}$}
& \multicolumn{1}{ c|}{$\ALG{Enc-KeyExp}$}
& \multicolumn{1}{ c|}{$\ALG{Dec-KeyExp}$}
& \multicolumn{1}{ c|}{.data} 
\\
\hline
\hline
%RV32IMC & Byte    &            &           &      312 &        0 &  522 \\
 RV32IMC & T-table &       804  &       804 &      154 &      174 & 5120 \\
 RV32IMC & \ISE{1} &       424  &       424 &\bftab 68 &        0 &   10 \\
 RV32IMC & \ISE{2} &\bftab 234  &\bftab 238 &\bftab 68 &       62 &   10 \\
 RV32IMC & \ISE{3} &       290  &       290 &       86 &       64 &   10 \\
 RV32IMC & \ISE{5} &       266  &       278 &      290 &        0 &   10 \\
\hline
 RV64IMC & \ISE{4} &       268  &       268 &      168 &      100 &    0 \\
\hline
\end{tabular}
\end{adjustbox}

\begin{adjustbox}{center,caption={Execution metrics
                                  for each ISE variant on the \CORE{2} core.
                                  Note that the $64$-bit \ISE{4} is absent, since there is no $64$-bit \CORE{2} core.
                                 },label={tab:eval:sw:perf:2},float={table}[!t]}
\centering
\begin{tabular}{|c|l|rr|rr|rr|rr|}
\hline
  \multicolumn{1}{|c|}{ISA}
& \multicolumn{1}{ c|}{Variant}
& \multicolumn{2}{ c|}{$\ALG{Enc}$}
& \multicolumn{2}{ c|}{$\ALG{Dec}$}
& \multicolumn{2}{ c|}{$\ALG{Enc-KeyExp}$}
& \multicolumn{2}{ c|}{$\ALG{Dec-KeyExp}$}
\\
\cline{3-10}
& / Goal
& \multicolumn{1}{ c|}{iret}
& \multicolumn{1}{ c|}{cycles}
& \multicolumn{1}{ c|}{iret}
& \multicolumn{1}{ c|}{cycles}
& \multicolumn{1}{ c|}{iret}
& \multicolumn{1}{ c|}{cycles}
& \multicolumn{1}{ c|}{iret}
& \multicolumn{1}{ c|}{cycles}
\\
\hline
\hline
%RV32IMC & Byte        &            &            &            &            &            &            &            &            \\
 RV32IMC & T-table     &          938 &         1016 &          938 &         1037&          430 &          515 &         1711 &         2307 \\ 
 RV32IMC & \ISE{1} (L) &          512 &          575 &          512 &          576& \bftab   198 & \bftab   302 & \bftab   204 & \bftab   321 \\
 RV32IMC & \ISE{1} (A) &          512 &          735 &          512 &          736& \bftab   198 &          342 & \bftab   204 &          361 \\
 RV32IMC & \ISE{2} (L) & \bftab   215 & \bftab   274 & \bftab   216 & \bftab   285& \bftab   198 & \bftab   302 &          335 &          615 \\
 RV32IMC & \ISE{2} (A) & \bftab   215 &          501 & \bftab   216 &          522& \bftab   198 &          332 &          335 &          753 \\
 RV32IMC & \ISE{3}     &          238 &          291 &          238 &          286&          219 &          312 &          659 &         1118 \\
 RV32IMC & \ISE{5} (L) &          227 &          294 &          227 &          291&          332 &          449 &          338 &          468 \\
 RV32IMC & \ISE{5} (A) &          227 &          554 &          227 &          532&          332 &          479 &          338 &          498 \\
\hline
\end{tabular}                
\end{adjustbox}

\begin{adjustbox}{center,caption={Execution metrics
                                  for each ISE variant on the \CORE{1} core.
                                  Note that the $64$-bit \ISE{4} uses the $64$-bit \CORE{1} core; all others use the $32$-bit \CORE{1} core.
                                 },label={tab:eval:sw:perf:1},float={table}[!t]}
\centering
\begin{tabular}{|c|l|rr|rr|rr|rr|}
\hline
  \multicolumn{1}{|c|}{ISA}
& \multicolumn{1}{ c|}{Variant}
& \multicolumn{2}{ c|}{$\ALG{Enc}$}
& \multicolumn{2}{ c|}{$\ALG{Dec}$}
& \multicolumn{2}{ c|}{$\ALG{Enc-KeyExp}$}
& \multicolumn{2}{ c|}{$\ALG{Dec-KeyExp}$}
\\
\cline{3-10}
& / Goal
& \multicolumn{1}{ c|}{iret}
& \multicolumn{1}{ c|}{cycles}
& \multicolumn{1}{ c|}{iret}
& \multicolumn{1}{ c|}{cycles}
& \multicolumn{1}{ c|}{iret}
& \multicolumn{1}{ c|}{cycles}
& \multicolumn{1}{ c|}{iret}
& \multicolumn{1}{ c|}{cycles}
\\
\hline
\hline
%RV32IMC & Byte        &            &            &            &            &            &            &            &            \\
 RV32IMC & T-table     &       934  &      1338  &       934  &      1003  &       430  &       569  &      1711  &      2167  \\
 RV32IMC & \ISE{1} (L) &       513  &       659  &       513  &       613  &\bftab 199  &       268  &\bftab 200  &\bftab 270  \\
 RV32IMC & \ISE{1} (A) &       513  &       791  &       513  &       725  &\bftab 199  &       308  &\bftab 200  &       310  \\
 RV32IMC & \ISE{2} (L) &\bftab 216  &\bftab 351  &\bftab 217  &       354  &\bftab 199  &\bftab 263  &       336  &       496  \\
 RV32IMC & \ISE{2} (A) &\bftab 216  &       503  &\bftab 217  &       534  &\bftab 199  &       293  &       336  &       637  \\
 RV32IMC & \ISE{3}     &       239  &       396  &       239  &       410  &       220  &       462  &       660  &      2420  \\
 RV32IMC & \ISE{5} (L) &       228  &       366  &       228  &\bftab 322  &       333  &       405  &       334  &       404  \\
 RV32IMC & \ISE{5} (A) &       228  &       536  &       228  &       546  &       333  &       438  &       334  &       434  \\
\hline
 RV64IMC & T-table     &       934  &      1086  &       934  &      1015  &       431  &       479  &      1712  &      1995  \\
 RV64IMC & \ISE{4}     &        76  &       115  &        76  &       133  &        61  &       199  &       131  &       286  \\
\hline
\end{tabular}
\end{adjustbox}

% -----------------------------------------------------------------------------

\begin{adjustbox}{center,caption={
    Comparison of performance/area product. Each value is normalised to the
    largest product per column. The {\tt RV64IMC} row is not normalised as
    there is no comparison point.
},label={tab:eval:results},float={table}[!t]}
\centering
\begin{tabular}{|c|l|rr|rr|rr|rr|}
\hline
  \multicolumn{1}{|c|}{ISA}
& \multicolumn{1}{ c|}{Variant}
& \multicolumn{2}{ c|}{$\ALG{Enc}$}
& \multicolumn{2}{ c|}{$\ALG{Dec}$}
& \multicolumn{2}{ c|}{$\ALG{EncKeyExp}$}
& \multicolumn{2}{ c|}{$\ALG{DecKeyExp}$}
\\
\cline{3-10}
& / Goal
& \multicolumn{1}{ c|}{iret}
& \multicolumn{1}{ c|}{cycles}
& \multicolumn{1}{ c|}{iret}
& \multicolumn{1}{ c|}{cycles}
& \multicolumn{1}{ c|}{iret}
& \multicolumn{1}{ c|}{cycles}
& \multicolumn{1}{ c|}{iret}
& \multicolumn{1}{ c|}{cycles}
\\
\hline
\hline
RV32IMC & \ISE{1} (L) &        1.00 &           1.00 &           1.00 &           1.00 &           0.50 &        0.57 &        0.51 &        0.51 \\
RV32IMC & \ISE{1} (A) &        0.62 &           0.80 &           0.62 &           0.80 &           0.31 &        0.40 & \bftab 0.32 & \bftab 0.36 \\
RV32IMC & \ISE{2} (L) &        0.43 &           0.48 &           0.43 &           0.50 &           0.51 &        0.58 &        0.85 &        1.00 \\
RV32IMC & \ISE{2} (A) &        0.16 &           0.34 &           0.16 &           0.35 &           0.19 &        0.24 & \bftab 0.32 &        0.46 \\
RV32IMC & \ISE{3}     & \bftab 0.15 & \bftab    0.17 & \bftab    0.15 & \bftab    0.16 & \bftab    0.18 & \bftab 0.19 &        0.54 &        0.58 \\
RV32IMC & \ISE{5} (L) &        0.53 &           0.61 &           0.53 &           0.60 &           1.00 &        1.00 &        1.00 &        0.89 \\
RV32IMC & \ISE{5} (A) &        0.22 &           0.47 &           0.22 &           0.45 &           0.41 &        0.44 &        0.41 &        0.39 \\
\hline
RV64IMC & \ISE{4}     &       0.266 &       0.402 &       0.266 &       0.465 &       0.213 &       0.696 &       0.458 &       1.000  \\
\hline
\end{tabular}
\end{adjustbox}

% =============================================================================
