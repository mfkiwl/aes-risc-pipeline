\documentclass[preprint]{iacrtrans}

\usepackage{paper}

\title{The design of scalar AES Instruction Set Extensions for RISC-V}
\keywords{ISE, AES, RISC-V}

\ifbool{anonymous}{%
\author{Anonymous Submission}
\institute{}
}{%
\author[
B. Marshall and
G. R. Newell and
D. Page and
M.-J. O. Saarinen and
C. Wolf
]{
Ben Marshall\inst{1}                \and
G. Richard Newell\inst{2}           \and
Dan Page\inst{1}                    \and
Markku-Juhani O. Saarinen\inst{3}   \and
Claire Wolf\inst{4}
}
\institute{
Department of Computer Science, University of Bristol \\ \email{{ben.marshall,daniel.page}@bristol.ac.uk}
\and
Microchip Technology Inc., USA \\ \email{richard.newell@microchip.com}
\and
PQShield, UK \\ \email{mjos@pqshield.com}
\and
Symbiotic EDA \\ \email{claire@symbioticeda.com}
}
}%

\begin{document}

% =============================================================================

\maketitle

\begin{abstract}
Secure, efficient execution of AES is an essential requirement on most
computing platforms. Dedicated
Instruction Set Extensions (ISEs) are often included for this purpose.
RISC-V is a (relatively) new ISA that lacks such a standardised ISE.
We survey the state-of-the-art industrial and academic ISEs for AES,
implement and evaluate five different ISEs, one of which is novel.
We recommend separate ISEs for 32 and 64-bit base architectures, with
measured performance improvements for an AES-128 block encryption of
$4\times$ 
and
$10\times$
with a hardware cost of $1.1K$ and $8.2K$ gates respectivley,
when compared to a software-only implementation based on use of T-tables.
We also explore how the proposed standard bit-manipulation extension
to RISC-V can be harnessed for efficient implementation of AES-GCM.
Our work supports the ongoing RISC-V cryptography extension standardisation
process.
\end{abstract}

% =============================================================================

\section{Introduction}
\label{sec:intro}
% =============================================================================

\paragraph{Implementing the Advanced Encryption Standard (AES).}

%
% Commented out since CHES folks know the histories.
%
%In October $2000$, NIST pronounced Rijndael~\cite{DaeRij:98,DaeRij:02}, 
%a design due to Daemen and Rijmen, 
%as winner of a $5$-year standardisation process~\cite{NBBBDFR:01} instigated 
%to identify a replacement for the incumbent
%Data     Encryption Standard (DES)~\cite{FIPS:46} 
%block cipher; the resulting 
%Advanced Encryption Standard (AES)~\cite{FIPS:197} 
%was announced in $2001$.

Compared to more general workloads, cryptographic algorithms like AES
present a significant implementation challenge.
They involve computationally intensive and specialised functionality,
are used in a wide range of contexts
and
form a central target in a complex attack surface.
The demand for efficiency (however measured) is an example of this
challenge in two ways.
First,
cryptography often represents an enabling technology vs. a feature and
is often viewed as an overhead from a user's perspective.
Addressing this is
complicated by constraints associated with the context, e.g., a demand 
for
high-volume, 
 low-latency, 
high-throughput, 
 low-footprint, 
and/or 
 low-power
 implementations.
Second,
although efficiency is a goal in itself, it {\em also} 
acts as an enabler for security.
This is because one {\em should not}
compromise security to meet efficiency requirements.
Hence a more efficient implementation leaves greater margin to deliver
countermeasures against an attack.

AES is an interesting case-study wrt. secure, efficient implementation.
For example,
per the request for candidates announcement,\footnote{%
\url{https://www.govinfo.gov/content/pkg/FR-1997-09-12/pdf/97-24214.pdf}
} the AES process was instrumental in popularising a model in which
{\em both}
``security''
(e.g., resilience against cryptanalytic attack)
{\em and}
``algorithm and implementation characteristics''
form important quality metrics for the {\em design}, in order to facilitate
techniques for higher quality {\em implementations} of it.
Additionally,
the design {\em and} implementations of AES are long-lived.
The importance of AES has led to special emphasis on related
research and development effort before, during, and, most significantly, 
after the AES process.
The $20+$ years since standardisation have forced an evolution of 
implementation techniques, to match changes in the technology 
and attack landscape.  For example,~\cite[Section 3.6]{NBBBDFR:01} covers
implementation (e.g., side-channel) attacks: this field has become richer,
and the associated threat more dangerous during said period.

% -----------------------------------------------------------------------------

\paragraph{Support via Instruction Set Extensions (ISEs).}

A large number of implementation styles often exist
for a given cryptographic algorithm.
Techniques can be
   algorithm-agnostic
   or
   algorithm-specific,
and based on the use of   
   hardware              only,
                software only,
   or
   a hybrid approach using ISEs ~\cite{GalBer:11,BarGioMar:09,RegIen:16}.
For the ISE case, the aim is to identify through benchmarking, pieces of
algorithm-specific functionality which are inefficiently represented in the
base ISA.
Said functions are then implemented in hardware, and exposed to the
programmer via one or more new instructions.

ISEs are an effective option for {\em both}
high-end, performance-oriented
and
 low-end, constrained
platforms. 
They are particularly effective for the latter where resource constraints
are tightest.
An ISE can be smaller and faster than a pure software implementation,
and more efficient in terms of performance gain per additional logic gate
than a hardware-only option.

Abstractly, an ISE design constitutes
an {\em interface} to domain-specific functionality through the
addition of instructions to a base ISA.
As a fundamental and long-lived computer systems interface, the design
and extension of an ISA demands careful consideration
(cf.~\cite[Section 4]{Gueron:09})
and the production of a concrete ISE design is not trivial.
It must deliver a quantified improvement to the workload in 
question {\em and}
consider numerous design goals including but not limited to:

\begin{itemize}
\item Limiting the number and complexity of changes and interactions with the
    parent ISA.
\item Avoiding the addition of too many instructions, or requiring large
    additional hardware modules to implement. This will hurt commercial
    adoption of the ISA.
\item Adhering to the design constraints and philosophies of the base ISA.
\item Maximising the utility of the additional functionality,
      i.e.,
      favour general-purpose over special-purpose functionality.
      Special-purpose functions can be justified in terms of how frequently
      the workload is required.
      For example, though an AES ISE might {\em only}
      be useful for AES, a webserver might execute AES millions of times
      per day.
\end{itemize}

\noindent
The x86 architecture provides many examples of ISE design,
having been extended numerous times by Intel and AMD.
Various generations of
non-cryptographic
Multi-Media      eXtensions (MMX),
Streaming SIMD  Extensions (SSE),
and
Advanced Vector Extensions (AVX)
support numerical algorithms via vector (or SIMD) vs. scalar computation.  
Likewise, the
    cryptographic
Advanced Encryption Standard New Instructions (AES-NI)~\cite{Gueron:09,DruGueKra:19}
ISE
supports AES: it significantly improves latency and throughput
(see, e.g.,~\cite{FazLopOli:18}),
and represents a useful case-study in the design goals above:
it adds just $6$ additional (vs. $1500+$ total) instructions,
reduces overhead by sharing the pre-existing XMM register file,
and facilitates compatibility via the
\VERB{CPUID}~\cite[Chapter 20]{X86:1:18}
feature identification mechanism.
It is also (sometimes unexpectedly) useful beyond AES:
the Gr{\o}stl hash function ~\cite{GKMMRST:11} uses the S-box,
and
the YAES~\cite{BosVer:14} authenticated encryption scheme uses a full round.
It can even be used to accelerate the Chinese SM4 block cipher.\footnote{\url{https://github.com/mjosaarinen/sm4ni}}

% =============================================================================

% =============================================================================

\paragraph{RISC-V.}
\label{sec:bg:riscv}

RISC-V is a (relatively) new ISA, with academic origins~\cite{riscv:1,riscv:2}.
Unlike x86 or ARMv8-A, RISC-V is a free-to-use 
open standard, managed by RISC-V International.
The base ISA is extremely simple, consisting of only $50$ instructions,
and adopts {\em strongly} RISC-oriented design principles.
RISC-V is also highly modular, having been {\em designed to be extended}.
The general-purpose base ISA can (optionally) be
supplemented using sets of special-purpose, standard or non-standard
extensions to
support additional functionality 
(e.g., floating-point, 
       via the 
       standard F~\cite[Section 11]{RV:ISA:I:19}
                and
                D~\cite[Section 12]{RV:ISA:I:19}
       extension),
or 
satisfy specific optimisation goals
(e.g., code density, 
       via the 
       standard C~\cite[Section 16]{RV:ISA:I:19}
       extension).
RISC-V International delegates the development of
extensions to a dedicated task group.
The Cryptographic Extensions Task
Group\footnote{
  \url{https://lists.riscv.org/g/tech-crypto-ext}
} provides some specific context for this paper, through their remit to 
develop scalar and vector extensions to support cryptography.

RISC-V uses $32$ registers,
denoted $\GPR[*][ i ]$ for $0 \leq i < 32$:
$
\GPR[*][ 0 ]
$
is fixed to $0$, whereas $\GPR[*][1]$ to $\GPR[*][31]$ are general-purpose.
\RVXLEN is used to denote the width of each $\GPR[*][ i ]$, and hence the base ISA.
We focus on extending the
RV32I~\cite[Section 2]{RV:ISA:I:19}
and 
RV64I~\cite[Section 5]{RV:ISA:I:19},
integer RISC-V base ISA
and therefore focus on systems where 
$\RVXLEN = 32$
or
$\RVXLEN = 64$.

% =============================================================================


\paragraph{Remit and organisation.}

In the context of an on-going effort to standardise cryptographic ISEs for
RISC-V, this paper investigates support for AES.
In specific terms, our contributions are as follows:

\begin{enumerate}

\item In 
      \REFSEC{sec:bg}
      we capture some background, including a limited
      Systematisation of Knowledge (SoK)
      for AES ISEs.

\item In 
      \REFSEC{sec:ise}
      we implement and evaluate five different ISEs for AES on two different 
      RISC-V CPU cores.
      We explore existing ISE designs, 
      and introduce what is, to the best of our knowledge, a novel ISE design
      in  \REFSEC{sec:ise:design:v5}
      that uses a quadrant-packed state representation.

\item In
      \REFSEC{sec:gcm}
      we evaluate how the
      proposed standard 
      Bit-manipulation
      extension~\cite[Section 21]{RV:ISA:I:19}
      to RISC-V can be used to efficiently implement AES-GCM.

\end{enumerate}

\noindent
On the one hand, 
RISC-V represents an excellent target for such work:
the ISA is extensible by design and its open nature makes
exploration of extensions easier through the availability of
(often open-source) implementations.  
Increased commercial deployment of such implementations suggests that work 
on RISC-V is timely and potentially of high impact.
On the other hand, RISC-V also presents unique challenges vs. previous work.
For example,
RISC-V could in fact be viewed as {\em three} related base ISAs,
 RV32I~\cite[Section 2]{RV:ISA:I:19},
 RV64I~\cite[Section 5]{RV:ISA:I:19},
and
RV128I~\cite[Section 6]{RV:ISA:I:19},
that each support a different word size:
designing ISEs that are applicable (or scale) across these options is a
complicating factor.
We hope this work supports RISC-V in becoming the
first widely implemented ISA to support AES acceleration across
all implementation profiles, from embedded IoT devices to application
and server class processors.

\ifbool{submission}{%
Note that in order to satisfy the TCHES submission guidelines, we have 
anonymised various resources and references.  We intend to open-source 
such resources post-submission, but could provide them to reviewers, 
if required, to facilitate the review process.
}{}%



% =============================================================================

\section{Background}
\label{sec:bg}

% -----------------------------------------------------------------------------

\subsection{AES specification}
\label{sec:bg:aes_spec}
% =============================================================================

% -----------------------------------------------------------------------------

\paragraph{Syntax.}

As a block cipher, AES defines two algorithms
\[
\begin{array}{lcl}
\ALG{Enc} &:& \SET{ 0, 1 }^{8 \cdot 4 \cdot Nk} \times \SET{ 0, 1 }^{8 \cdot 4 \cdot Nb} \rightarrow \SET{ 0, 1 }^{8 \cdot 4 \cdot Nb} \\
\ALG{Dec} &:& \SET{ 0, 1 }^{8 \cdot 4 \cdot Nk} \times \SET{ 0, 1 }^{8 \cdot 4 \cdot Nb} \rightarrow \SET{ 0, 1 }^{8 \cdot 4 \cdot Nb} \\
\end{array}
\]
st.
$
m = \ALG{Dec}( k, c = \ALG{Enc}( k, m ) ) .
$
That is, given a plaintext $m$ and cipher key $k$, \ALG{Enc} encrypts $m$ 
under $k$; given the same $k$, \ALG{Dec} will invert \ALG{Enc} and so the
{\em same} $m$ can be recovered from the associated ciphertext $c$.  
In addition, it defines an algorithm
\ALG{KeyExp}
that expands~\cite[Section 5.2]{FIPS:197} the cipher key into a sequence 
of round keys then used by
\ALG{Enc}
or
\ALG{Dec};
where appropriate, we use
\[
\begin{array}{lcl}
\ALG{Enc-KeyExp} &:& \SET{ 0, 1 }^{8 \cdot 4 \cdot Nk} \rightarrow \SET{ 0, 1 }^{( 8 \cdot 4 \cdot Nb ) \times ( Nr + 1 )} \\
\ALG{Dec-KeyExp} &:& \SET{ 0, 1 }^{8 \cdot 4 \cdot Nk} \rightarrow \SET{ 0, 1 }^{( 8 \cdot 4 \cdot Nb ) \times ( Nr + 1 )} \\
\end{array}
\]
to denote said algorithm as specialised to suit
\ALG{Enc}
and
\ALG{Dec}
respectively.

% -----------------------------------------------------------------------------

\paragraph{Parameterisation.}

An AES parameter set~\cite[Figure 4]{FIPS:197}
is a triple
$
\TUPLE{ Nk, Nb, Nr }
$
where 
$Nk$ dictates the number of $32$-bit words in $k$,
$Nb$ dictates the number of $32$-bit words in $m$ or $c$ (i.e., a block),
and
$Nr$ dictates the number of rounds.  The standard AES parameter sets are
\[
\begin{array}{lcl}
\mbox{AES-128} &\mapsto& \TUPLE{ 4, 4, 10 } \\
\mbox{AES-192} &\mapsto& \TUPLE{ 6, 4, 12 } \\
\mbox{AES-256} &\mapsto& \TUPLE{ 8, 4, 14 } \\
\end{array}
\]
st. the number of bits in a plaintext (resp. ciphertext) block is fixed to 
$
8 \cdot 4 \cdot Nb = 128 .
$
From here on, we focus wlog. on encryption using AES-128 (other parameter 
sets are catered for naturally, and decryption with minor differences) so
use the terms AES and AES-128 synonymously.

% -----------------------------------------------------------------------------

\paragraph{Design.}

The mathematics underpinning AES are described in ~\cite[Section 4]{FIPS:197}.
In particular, it can be defined in terms of 
operations in the finite field $\F_{2^{  8}}$ constructed as
$
\F_{2}[\IND{x}] / ( \IND{x}^{8} + \IND{x}^{4} + \IND{x}^{3} + \IND{x} + 1 ) .
$
A hexadecimal short-hand~\cite[Section 3.2]{FIPS:197} is used to represent 
field literals, e.g.,
$
\AESCONST{13} ~\mapsto~ \RADIX{13}{16} ~\equiv~ \RADIX{00010011}{2} ~\mapsto~ \IND{x}^4 + \IND{x} + 1 .
$
Field 
      addition, 
multiplication, 
and  
      division
are denoted by
$\AESADD$,
$\AESMUL$,
and
$\AESINV$
respectively,
with the multiplication-by-$\IND{x}$ operation~\cite[Section 4.2.1]{FIPS:197} 
denoted \AESFUNC{xtime}.
Elements of $\F_{2^8}$ are collected into $( 4 \times 4 )$-element state
and round key matrices; the $i$-th row and $j$-th column of such a matrix 
relating to round $r$ is denoted
$\AESRND {s}{r}_{i,j}$
and
$\AESRND{rk}{r}_{i,j}$
respectively, with super- and/or subscripts omitted whenever irrelevant.

AES is an iterative block cipher, based on a substitution-permutation network.
This means encryption using AES can be described~\cite[Section 5.2]{FIPS:197}
as follows:
1)    the  input  plaintext is pre-whitened to yield
      $\AESRND {s}{  0} = m \AESADD \AESRND{rk}{0} = m \AESADD k$,
2)    each $r$-th round, for $1 \leq r \leq Nr$, demands computation of
      $\AESRND {s}{r+1} = \ALG{P-layer}( \ALG{S-layer}( \AESRND{s}{r}                        ) ) \AESADD \AESRND{rk}{r}$,
      and therefore use of round key
      $\AESRND{rk}{r  }$,
3)    the output ciphertext is
      $c = \AESRND{s}{Nr}$.
Note that an alternative round definition, namely
      $\AESRND {s}{r+1} = \ALG{P-layer}( \ALG{S-layer}( \AESRND{s}{r} \AESADD \AESRND{rk}{r} ) )                       $ ,
is plausible: this shifts the 
 pre-whitening step {\em before} 2) 
into an analogous 
post-whitening step {\em  after} 2)
to yield an equivalent result.
At a  low(er) level,
the computation of each round is specified via four round functions (each of 
which has an inverse, to support decryption):

\begin{itemize}

\item \AESFUNC{SubBytes}
      ~\cite[Section 5.1.1]{FIPS:197}
      operates element-wise,
      computing
      $\AESRND{s}{r+1}_{i,j} = \ALG{S-box}( \AESRND{s}{r}_{i,j} )$
      via application of the S-box:
      given an element $x$, this component can be described as
      \[
      \begin{array}{lcl}
      \ALG{S-Box} &:& \left\{\begin{array}{ccc}
                             \F_{2^8} &\rightarrow& \F_{2^8} \\
                             x        &\mapsto    & f(g(x))  \\
                             \end{array}
                      \right.
      \end{array}
      \]
      where 
      $g$ is an inversion, 
      and 
      $f$ is a specially selected affine transformation.
      Where appropriate,
      we overload \AESFUNC{SubBytes} by allowing it to denote application 
      of the S-box to {\em any} collection, 
      e.g., a row, column, or, more generally, a sequence, 
      of elements.

\item \AESFUNC{ShiftRows}
      ~\cite[Section 5.1.2]{FIPS:197}
      operates     row-wise,
      rotating each 
      $i$-th row 
      of 
      $\AESRND{s}{r  }$
      by $i$ elements
      to form 
      the associated row    of
      $\AESRND{s}{r+1}$,
      i.e.,
      $\AESRND{s}{r+1}_{i,j} = \AESRND{s}{r}_{i,j + i \pmod{Nb}}$.
      Where appropriate,
      we use
      \AESFUNC{ShiftRow}
      to denote
      the operation applied to a single 
      row
      within \AESFUNC{ShiftRows}.

\item \AESFUNC{MixColumns}
      ~\cite[Section 5.1.3]{FIPS:197}
      operates  column-wise,
      multiplying each 
      $j$-th column
      of 
      $\AESRND{s}{r  }$
      with a constant MDS matrix
      to form 
      the associated column of
      $\AESRND{s}{r+1}$.
      Where appropriate,
      we use
      \AESFUNC{MixColumn}
      to denote
      the operation applied to a single 
      column 
      within \AESFUNC{MixColumns}, i.e., multiplication of a $4$-element 
      column vector by the constant MDS matrix.
      
\item \AESFUNC{AddRoundKey}
      ~\cite[Section 5.1.4]{FIPS:197}
      operates element-wise,
      computing
      $\AESRND{s}{r+1}_{i,j} = \AESRND{s}{r}_{i,j} \AESADD \AESRND{rk}{r}_{i,j}$ 
      and thereby mixing a round key into the state.

\end{itemize}

\noindent
Note that
$
\ALG{S-layer} = \AESFUNC{SubBytes} ,
$
and
\[
\ALG{P-layer} = \left\{\begin{array}{l@{\;}c@{\;}l lr}
                       \AESFUNC{MixColumns} &\circ& \AESFUNC{ShiftRows} & \mbox{in rounds} & 1 \leq r < Nr \\
                                            &     & \AESFUNC{ShiftRows} & \mbox{in round } &            Nr \\
                       \end{array}
                \right.
\]
i.e., the last, $Nr$-th round differs from the initial $Nr - 1$ rounds.  As
such, a round as defined above is constructed via
$
\AESFUNC{AddRoundKey} \circ \AESFUNC{MixColumns} \circ \AESFUNC{ShiftRows} \circ \AESFUNC{SubBytes} 
$
or
$
\AESFUNC{AddRoundKey} \circ                            \AESFUNC{ShiftRows} \circ \AESFUNC{SubBytes}
$
respectively, where, because \AESFUNC{ShiftRows} and \AESFUNC{SubBytes}
commute, the order they are applied in can be selected to suit.



\subsection{AES implementation}
\label{sec:bg:aes_impl}

\subsubsection{Representation}
\label{sec:bg:aes_impl_rep}
% =============================================================================

A field element in $\F_{2^8}$ can be represented by an
$8$-bit byte,
where the $i$-th bit of $x$ for $0 \leq i < 8$ represents the $i$-th 
polynomial coefficient.

Beyond this, the state and round key matrices can be represented in
several ways.
The most direct option would be termed
array-based (or unpacked):
the matrix is represented as a $16$-element array of $8$-bit bytes, each
representing field elements.
%FIPS-197~\cite{FIPS:197} defines a word to be st. $w = 32$.
We use $R$ to refer to the register width of a target platform.
For RISC-V, $R = \RVXLEN$ where we consider $\RVXLEN \in {32,64}$.
Where $R \geq  32$,
an entire row or column of the AES state matrix can be packed into each 
register:
we term these
   ``row-packed''  
and
``column-packed''
representations respectively.
Where $R \geq 128$, 
it is plausible to pack
an entire AES state matrix
into a single register: 
we term this a 
 ``fully-packed'' 
representation.

% =============================================================================


\subsubsection{Hardware-only implementations}
\label{sec:bg:aes_impl_hw}
% =============================================================================

In a hardware-only implementation,
execution of 
AES
is 
performed by 
a dedicated hardware module (e.g., a memory-mapped co-processor),
while
the software which uses AES is executed on
a general-purpose CPU core.
A large design space exists for hardware implementations of AES.
Gaj and Chodowiec~\cite[Section 3.3]{GajCho:00}
give an overview, detailing
iterative,
combinatorial (unrolled),
and
pipelined architectures.
Similarly, ~\cite{PMDW:04,GooBen:05,GajCho:09}
survey concrete implementations on a variety of fabrics including FPGAs
and ASICs.

Although hardware-only designs are not our focus, the associated techniques
can guide ISE-related design choices.
First,
they guide the ISE interface.
For example, some ISEs can be characterised as offering an interface to
hardware constituting one round 
(i.e., aligned with an iterative hardware implementation).
Second,
they guide the ISE implementation.
For example, a significant body of work focuses on efficient hardware 
implementation of the S-box: ~\cite{Canright:05,BoyPer:12,ReyTahAsh:18}.

% =============================================================================

\subsubsection{Software-only implementations}
\label{sec:bg:aes_impl_sw}
% =============================================================================

In a software-only implementation,
execution of AES
and
the associated application program
is 
performed by 
a general-purpose processor core, using only instructions in the base ISA.
Since we only consider use of the RISC-V scalar base ISA, we exclude work on
the use of vector-like extensions~\cite{Hamburg:09}.

Software-only techniques are important because 
many ISEs are evaluated against baseline ISA implementations.
Work such as that of
Bernstein and Schwabe~\cite{BerSch:08},
Osvik et al.~\cite{OBSC:10},
and
Schwabe and Stoffelen~\cite{SchSto:16}
present and compare multiple techniques across a range of platforms, but,
for completeness, we present a (limited) survey in what follows.

% -----------------------------------------------------------------------------

\paragraph{Compute-oriented.}

A compute-oriented implementation of AES favours
 online     computation, 
thus reducing 
memory footprint
at the cost of increased 
latency.
Following~\cite[Section 4.1]{DaeRij:02}, for example, the idea is to simply
1) adopt an
    array-packed
   representation of state and round key matrices,
   then
2) construct a round implementation by following the algorithmic description
   of each round function in a direct manner.
Addition in $\F_{2^8}$ can be implemented with a base ISA XOR instruction.
Base ISA support is rarely present for multiplication and inversion in
$\F_{2^8}$ however.
Hence it is common to pre-compute the \ALG{S-box} and/or \AESFUNC{xtime} 
functions.
This requires pre-computation and storage of a
$
\SI{256}{\byte}
$
look-up table per function, but significantly reduces execution latency.

On platforms where $R = 32$,
Bertoni et al.~\cite{BBFMM:02}
improve execution latency by exploiting the wider data-path.
They adopt a row-packed representation of state and round key matrices,
implementing
\AESFUNC{ShiftRows}
using native rotation instructions to act on the packed rows.
\AESFUNC{MixColumns} is implemented
using the SIMD Within A Register (SWAR) paradigm:
applying
\AESFUNC{xtime}
across a packed row in parallel.

% -----------------------------------------------------------------------------

\paragraph  {Table-oriented.}

A  table-oriented implementation of AES favours
offline pre-computation,
reducing 
latency
but increasing the
memory footprint.
The main example of this technique is the so-called
T-tables~\cite[Section 4.2]{DaeRij:02} method.
This involves
adopting a 
column-packed
representation of state and round key matrices and
pre-computing
   $
   \AESFUNC{MixColumn} \circ \AESFUNC{SubBytes}
   $
   using the tables
   \[
   \begin{array}{cc}
   \begin{array}{lcl}
   T_0[x] &=& \left[\begin{array}{c}
                    \RADIX{02}{16} \AESMUL \ALG{S-box}( x ) \\
                    \RADIX{01}{16} \AESMUL \ALG{S-box}( x ) \\
                    \RADIX{01}{16} \AESMUL \ALG{S-box}( x ) \\
                    \RADIX{03}{16} \AESMUL \ALG{S-box}( x ) \\
                    \end{array} \right]
   \end{array}
   &
   \begin{array}{lcl}
   T_1[x] &=& \left[\begin{array}{c}
                    \RADIX{03}{16} \AESMUL \ALG{S-box}( x ) \\
                    \RADIX{02}{16} \AESMUL \ALG{S-box}( x ) \\
                    \RADIX{01}{16} \AESMUL \ALG{S-box}( x ) \\
                    \RADIX{01}{16} \AESMUL \ALG{S-box}( x ) \\
                    \end{array} \right]
   \end{array}
   \\\\
   \begin{array}{lcl}
   T_2[x] &=& \left[\begin{array}{c}
                    \RADIX{01}{16} \AESMUL \ALG{S-box}( x ) \\
                    \RADIX{03}{16} \AESMUL \ALG{S-box}( x ) \\
                    \RADIX{02}{16} \AESMUL \ALG{S-box}( x ) \\
                    \RADIX{01}{16} \AESMUL \ALG{S-box}( x ) \\
                    \end{array} \right]                 
   \end{array}
   &
   \begin{array}{lcl}
   T_3[x] &=& \left[\begin{array}{c}
                    \RADIX{01}{16} \AESMUL \ALG{S-box}( x ) \\
                    \RADIX{01}{16} \AESMUL \ALG{S-box}( x ) \\
                    \RADIX{03}{16} \AESMUL \ALG{S-box}( x ) \\
                    \RADIX{02}{16} \AESMUL \ALG{S-box}( x ) \\
                    \end{array} \right]
   \end{array}
   \end{array}
   \]
   for $x \in \F_{2^8}$,
   then computing each $j$-th column of $\AESRND{s}{r+1}$ as
   \[
   T_0[ \AESRND{s}{r}_{i, j + i \pmod{Nb}} ] \AESADD
   T_1[ \AESRND{s}{r}_{i, j + i \pmod{Nb}} ] \AESADD
   T_2[ \AESRND{s}{r}_{i, j + i \pmod{Nb}} ] \AESADD
   T_3[ \AESRND{s}{r}_{i, j + i \pmod{Nb}} ]
   \]
   where extraction of elements caters for \AESFUNC{ShiftRows}, then XOR'ing 
   the $j$-th column of $\AESRND{rk}{r}$ to cater for \AESFUNC{AddRoundKey}.

As such, each round becomes a sequence of look-ups into $T_i$, plus XORs 
to combine their result.
Doing so demands pre-computation and storage of a
$
256 \cdot \SI{4}{\byte} = \SI{1}{\kilo\byte}
$
look-up table per $T_i$.
The overhead related to extraction of each element from 
packed columns representing $\AESRND{s}{r}$ 
(to form look-table offsets) 
can be significant:
Fiskiran and Lee~\cite{FisLee:01}
analyse the impact of different addressing modes on this issue, with
Stoffelen~\cite[Section 3.1]{Stoffelen:19}
concluding that RISC-V is ill-equipped to reduce said overhead,
due to the provision of a sparse set of addressing modes.
Futher, in systems with data caches, T-table based implementations are
succeptible to timing attacks \cite{DJB:05}.

% -----------------------------------------------------------------------------

\paragraph{Bit-sliced.}

The term bit-slicing is an implementation technique due to
Biham~\cite{Biham:97},
which constitutes

\begin{enumerate}
\item a non-standard {\em representation}
      of data where
      each $R$-bit word $x$ is transformed into $\REP{x}$,
      i.e.,
      $R$ slices, say $\REP{x}[ i ]$ for
      $
      0 \leq i < R ,
      $
      where $\REP{x}[ i ]_j = x_i$ for some $j$,
      and
\item a non-standard {\em implementation}
      of operation:
      each operation $f$ used as
      $
          {r} =     {f}(     {x} )
      $
      must be transformed into a ``software circuit'' $\REP{f}$,
      i.e.,
      a sequence of Boolean instructions acting on the slices st.
      $
      \REP{r} = \REP{f}( \REP{x} ) .
      $
\end{enumerate}

\noindent
Bit-slicing introduces some overhead related to conversion of $x$ into
$\REP{x}$ and $\REP{r}$ into $r$, plus the (relative) inefficiency
of $\REP{f}$ vs. $f$ wrt. latency and footprint.
However, if each slice is itself an $R$-bit word, then it
is possible to compute $R$ instances of $\REP{f}$ in {\em parallel}
on suitably packed $\REP{x}$.
A common analogy is that of transforming the 
$R$-bit, $1$-way scalar processor 
into a 
$1$-bit, $R$-way SIMD   processor, 
thus giving (or recouping) up to a $R$-fold improvement in latency.

As evidenced by \cite{MatNak:07,Konighofer:08} and \cite{KasSch:09},
the application of bit-slicing to AES can be very effective;
Stoffelen~\cite[Section 3.1]{Stoffelen:19}
specifically investigates this fact within the context of RISC-V.

% =============================================================================

\subsection{Existing AES ISEs}
\label{sec:bg:aes_impl_ise}
% =============================================================================

Here, we survey AES-related ISE designs split into
1) industry-specified ISEs,
   which are {\em     standard} extensions,
   and
2) academia-specified ISEs,
   which are {\em non-standard} extensions,
wrt. a given base ISA.
   Each ISE is classified as either
   workload-specific,
   if it is only useful for AES,
   or
   workload-agnostic,
   if it is      useful for AES and other workloads.
Note that we exclude work where an ISE for another workload can be applied 
{\em  to} AES
but was not designed 
{\em for} AES
(see, e.g., Tillich and Gro{\ss}sch\"{a}dl~\cite{TilGro:04} who apply an ISE intended for ECC to AES).

% =============================================================================

\subsubsection{Standard, industry-specified ISEs.}

\paragraph{Intel.}
      Introduced support for AES in 
      x86
      per~\cite[Section 12.13]{X86:1:18}.
      Instructions use a
          destructive $2$-address ($1$ source, $1$ source/destination)  
      or
      non-destructive $3$-address ($2$ source, $1$        destination)
      format
      depending on the variant (e.g., XMM- vs. AVX-based),
      and operate on data housed in the pre-existing
      vector 
      register file, implying $R = 128$.
      AES is implemented by
      1) adopting a 
          fully-packed
         representation of state and round key matrices,
         then
      2) using
             \VERB{AESENC}         ~\cite[Page 3-54]{X86:2:18}
         to construct a round implementation as
         \[
         \VERB{AESENC} \mapsto \AESFUNC{AddRoundKey} \circ \AESFUNC{MixColumns} \circ \AESFUNC{SubBytes} \circ \AESFUNC{ShiftRows}
         \]
%     Note that
%            \VERB{AESENCLAST}     ~\cite[Page 3-56]{X86:2:18}
%     supports 
%     the $Nr$-th round;
%     additional instructions are provided to 
%     support
%     decryption
%     (i.e., \VERB{AESDEC}         ~\cite[Page 3-50]{X86:2:18}
%            and
%            \VERB{AESDECLAST}     ~\cite[Page 3-52]{X86:2:18})
%     and
%     key expansion
%     (i.e., \VERB{AESKEYGENASSIST}~\cite[Page 3-59]{X86:2:18}
%            and
%            \VERB{AESIMC}         ~\cite[Page 3-58]{X86:2:18}).

\paragraph {IBM.}
      Introduced support for AES in 
      POWER
      per~\cite[Section 6.11.1]{POWER:18}.
      Instructions use a
      non-destructive $3$-address ($2$ source, $1$        destination)
      format,
      and operate on data housed in the pre-existing
      vector 
      register file, implying $R = 128$.
      AES is implemented by
      1) adopting a 
          fully-packed
         representation of state and round key matrices,
         then
      2) using
             \VERB{vcipher}     ~\cite[Page 304]{POWER:18}
         to construct a round implementation as
         \[
         \VERB{vcipher} \mapsto \AESFUNC{AddRoundKey} \circ \AESFUNC{MixColumns} \circ \AESFUNC{ShiftRows} \circ \AESFUNC{SubBytes}
         \]
%     Note that
%            \VERB{vcipherlast} ~\cite[Page 304]{POWER:18}
%     supports 
%     the $Nr$-th round;
%     additional instructions are provided to 
%     support
%     decryption
%     (i.e., \VERB{vncipher}    ~\cite[Page 305]{POWER:18}
%            and
%            \VERB{vncipherlast}~\cite[Page 305]{POWER:18})
%     and
%     key expansion
%     (i.e., \VERB{vsbox}       ~\cite[Page 305]{POWER:18}).

\paragraph{ARM.}
      Introduced support for AES in 
      ARMv8-A
      per~\cite[Section A2.3]{ARMv8-A:20}.
      Instructions use a
          destructive $2$-address ($1$ source, $1$ source/destination)  
      format,
      and operate on data housed in the pre-existing
      vector 
      register file, implying $R = 128$.
      AES is implemented by
      1) adopting a 
          fully-packed
         representation of state and round key matrices,
         then
      2) using
             \VERB{AESE}  ~\cite[Section C7.2.8 ]{ARMv8-A:20}
             and
             \VERB{AESMC} ~\cite[Section C7.2.10]{ARMv8-A:20}
         to construct a round implementation as
         \[
         \VERB{AESMC} \circ \VERB{AESE} \mapsto \AESFUNC{MixColumns} \circ ( \AESFUNC{SubBytes} \circ \AESFUNC{ShiftRows} \circ \AESFUNC{AddRoundKey} ) ,
         \]
%         where the alternative round definition from 
%         \REFSEC{sec:bg:aes_spec} 
%         is assumed to cater for the order of application.
%     Note that
%     additional instructions are provided to 
%     support
%     decryption
%     (i.e., \VERB{AESD}  ~\cite[Section C7.2.7 ]{ARMv8-A:20}
%            and
%            \VERB{AESIMC}~\cite[Section C7.2.9 ]{ARMv8-A:20}),
%     but none are required to 
%     support
%     the $Nr$-th round:
%     \VERB{AESE} obviously lacks \AESFUNC{MixColumns}, and the post-whitening 
%     step is naturally supported via XOR. 

\paragraph{Oracle.}
      Introduced support for AES in 
      SPARC 
      per~\cite[Sections 7.3+7.4]{SPARC:16}.
      Instructions use a
      non-destructive $4$-address ($3$ source, $1$        destination)
      format,
      and operate on data housed in the pre-existing
      general-purpose
      register file, implying $R =  64$.
      AES is implemented by
      1) using a 
         column-packed
         representation of state and round key matrices,
         then
      2) using
             \VERB{AES_EROUND01}     ~\cite[Page 109]{SPARC:16}
             and
             \VERB{AES_EROUND23}     ~\cite[Page 109]{SPARC:16}
         to construct a round implementation as
         \[
         ( \VERB{AES_EROUND01};\VERB{AES_EROUND23} ) \mapsto \AESFUNC{AddRoundKey} \circ \AESFUNC{MixColumns} \circ \AESFUNC{ShiftRows} \circ \AESFUNC{SubBytes} 
         \]
         in two steps:
         the first  step processes columns $0$ and $1$ via \VERB{AES_EROUND01}
         whereas
         the second step processes columns $2$ and $3$ via \VERB{AES_EROUND23}.
%     Note that
%            \VERB{AES_EROUND01_LAST}~\cite[Page 109]{SPARC:16}
%            and
%            \VERB{AES_EROUND23_LAST}~\cite[Page 109]{SPARC:16}
%     support 
%     the $Nr$-th round;
%     additional instructions are provided to 
%     support
%     decryption
%     (i.e., \VERB{AES_DROUND01}     ~\cite[Page 109]{SPARC:16},
%            \VERB{AES_DROUND23}     ~\cite[Page 109]{SPARC:16},
%            \VERB{AES_DROUND01_LAST}~\cite[Page 109]{SPARC:16},
%            and
%            \VERB{AES_DROUND23_LAST}~\cite[Page 109]{SPARC:16})
%     and
%     key expansion
%     (i.e., \VERB{AES_KEXPAND0}     ~\cite[Page 112]{SPARC:16},
%            \VERB{AES_KEXPAND1}     ~\cite[Page 109]{SPARC:16},
%            and
%            \VERB{AES_KEXPAND2}     ~\cite[Page 112]{SPARC:16}).

% -----------------------------------------------------------------------------

\subsubsection{Non-standard, academia-specified ISEs.}

% workload-agnostic

      Burke et al.~\cite{BurMcDAus:00}
      propose 
      a workload-agnostic ISE
      based on workload characterisation.
      Per~\cite{BurMcDAus:00}, pertinent examples
      for AES
      include
      a) \VERB{ROL}
         and
         \VERB{ROR},
         which perform
         left- and right-rotate,
         and
      b) \VERB{SBOX},
         which 
         extracts elements to form look-up table offsets.
         In one configuration,
         the resulting memory accesses are supported by a
         set of special-purpose ``S-box caches''.

      Fiskiran and Lee~\cite{FisLee:05}
      propose 
      a workload-agnostic ISE
      that employs a so-called
      Parallel Table Lookup Module (PTLU).
      For AES, 
      this accelerates implementations based on T-tables 
      by affording an addressing mode that
      a) integrates 
         extraction of elements to form look-up table offsets,
         and
      b) performs the associated table look-ups in parallel, supported by
         a dedicated scratch-pad memory.

      Biham et al.~\cite[Page 232]{BihAndKnu:98}
      propose (in theory)
      and
      Grabher et al.~\cite{GraGroPag:08}
      explore  (in practice)
      a workload-agnostic ISE
      that supports bit-sliced implementations.
      The ISE allows computation using 
      {\em configurable} $4$-input, $2$-output 
      Boolean functions, vs. 
      {\em fixed}        $2$-input, $1$-output alternatives such as NOT, AND, OR, and XOR.
      Sequences of native Boolean instructions, which dominate bit-sliced
      implementations, can thereby be ``compressed'' into use of the ISE.
      Doing so improves both latency and footprint.
      \cite[Section 4]{GraGroPag:08} details the application to AES.

% workload-specific

      Nadehara et al.~\cite{NadIkeKur:04} 
      propose 
      a workload-specific ISE
       that could be described as 
      ``hardware-assisted T-tables'':
      observing that $\forall x, i \neq j$, $T_i[ x ]$ is a rotation of
      $T_j[ x ]$, they support on-the-fly computation (vs. via look-up)
      of T-table entries.
      The ISE constitutes a single instruction
      $\VERB{AESENC} \mapsto T_i$,
      supported by a dedicated hardware module
      (see~\cite[Figure 6]{NadIkeKur:04}).
      Instances of \VERB{AESENC}
      1) extract an   input element from a 
         packed  input column
      2) use the input to compute an output element equivalent to a
         look-up from the T-table,
         and
      3) store   the output element into a
         packed output column.
      This approach was reapplied by Saarinen~\cite{Saarinen:20}
      within the context of RISC-V.

      Tillich et al.~\cite{TilGroSze:05}
      propose 
      a workload-specific ISE
       that could be described as 
      ``hardware-assisted S-box''.
      The ISE constitutes a single instruction
      $\VERB{sbox} \mapsto \AESFUNC{SubBytes}$,
      supported by a dedicated hardware module
      (see~\cite[Figure 1]{TilGroSze:05}).
      Instances of \VERB{sbox}
      1) extract an   input element from a packed  input row or column,
      2) use the input to compute an output element equivalent to a
         look-up from the S-box,
         and
      3)  insert the output element into a packed output row or column.
         Using insert vs. overwrite semantics allows
         \AESFUNC{ShiftRows} to be computed {\em for free}.

      Bertoni et al.~\cite{BBFR:06}
      propose 
      a workload-specific ISE
       that could be described as 
      ``hardware-assisted round functions''.
      The ISE includes
      1) zero-overhead rotation (similar to ARM),
         and
      2) byte- and word-oriented variants of
         $\VERB{SMix} \mapsto \AESFUNC{MixColumn} \circ \AESFUNC{SubBytes}$.
      
      Tillich and Gro{\ss}sch\"{a}dl~\cite{TilGro:06}
      propose 
      a workload-specific ISE
       that could be described as 
      ``hardware-assisted round functions''.
      The ISE includes
         byte- and word-oriented variants of
         $\VERB  {sbox[4][s|r]} \mapsto \AESFUNC{SubBytes} $
         and
         $\VERB{mixcol[4][s]  } \mapsto \AESFUNC{MixColumn}$;
      per~\cite[Section 4.3]{TilGro:06},
      the most efficient variant allows
         a zero-overhead implementation of \AESFUNC{ShiftRows} to be realised.


% =============================================================================


\subsection{Security}
\label{sec:bg:aes_impl_sec}
% =============================================================================

While the security of AES against a cryptanalytic attack is defined by
the design, and so is out of scope, {\em implementation} attacks are
of central importance.
An implementation attack focuses on the concrete instance of a construct
rather than the abstract specification.
Countermeasures against such attacks must therefore be
considered alongside implementations they relate to.
Since AES is an important target, a significant body of literature exists
around implementation attacks on it, including both
 active (e.g., fault injection)
or
passive (i.e., side-channel monitoring)
attack techniques.
The latter can be sub-divided into those dependent on
analogue
(power-based~\cite{ManOswPop:07})
or
discrete 
(time-based~\cite{KoeQui:99})
leakage.

Use of ISEs
{\em can} provide some inherent protection against certain attacks.
For example,
ISEs typically yield constant time execution,
preventing some classes of timing or micro-architectural
attack techniques
(see~\cite[Section 4]{Szefer:19} and~\cite[Section 4]{GYCH:18}).
Unfortunately,
use of ISEs also presents some unique challenges.
For example, 
Saab et al. ~\cite{SaaRohHam:16}
discuss power-based attacks on AES-NI; concluding
that naive use of AES-NI yields exploitable information leakage.
Mitigation of such leakage demands the ISE
address instances where the leakage stems from ``inside'' the ISE,
and work with appropriate countermeasures
(e.g., hiding~\cite[Chapter 7]{ManOswPop:07} or masking~\cite[Chapter 10]{ManOswPop:07}).
Tillich et al.~\cite{TilHerMan:07}
consider this problem to an extent, including an ISE-based option in
their investigation of hardened AES implementations. However, the challenge
of developing suitable ISEs is under-studied in general.

% =============================================================================


% =============================================================================

\section{Exploring AES ISEs for RISC-V}
\label{sec:ise}

% -----------------------------------------------------------------------------

\label{sec:ise:design}

% =============================================================================

\REFSEC{sec:bg:aes_impl_ise}
outlined a range of ISE designs, demonstrating a large design space of
options that we {\em could} consider.  To narrow the design space into
those we {\em do} consider, we use the requirements outlined below:

\begin{requirement}\label{req:1}
The ISE must support
1) AES encryption {\em and} decryption,
   and
2) {\em all} parameter sets, i.e., AES-128, AES-192, and AES-256.
Support for 
auxiliary operations, e.g., key schedule, 
is an advantage but not a requirement.
\end{requirement}

\begin{requirement}\label{req:2}
The ISE must align with the wider RISC-V design principles.
This means it should 
favour simple building-block operations,
and
use instruction encodings with at most
$2$ source registers and
$1$ destination register.
This avoids the cost of a general-purpose register file with more than $2$
read ports or $1$ write port.
\end{requirement}

\begin{requirement}\label{req:3}
The ISE must use
the RISC-V general-purpose scalar register file 
to store operands and results, rather than
any vector register file.
This requirement excludes the majority of standard ISEs outlined in 
\REFSEC{sec:bg:aes_impl_ise}.
\end{requirement}

\begin{requirement}\label{req:4}
The ISE must not introduce
special-purpose       architectural state, 
nor rely on
special-purpose micro-architectural state
(e.g., caches or scratch-pad memory).
\end{requirement}

\begin{requirement}\label{req:5}
The ISE must enable data-oblivious execution of AES, preventing
timing attacks based on execution latency
(e.g., stemming from accesses to a pre-computed S-box).
\end{requirement}

\begin{requirement}
The ISE must be efficient, in terms of improvement in execution latency 
per area required: this balances the value in {\em both} metrics vs. an 
exclusive preference for one or the other.
Efficiency wrt. 
auxiliary metrics, e.g., memory footprint or instruction encoding points,
is an advantage but not a requirement.
\end{requirement}

\noindent
Overall, the requirements combine to intentionally target the ISE at 
 low(er)-end,
resource-constrained (e.g., embedded) platforms.  
We view such a focus as reasonable, because existing work on adding
cryptographic support to the
standard 
vector extension ~\cite[Section 21]{RV:ISA:I:19}
already caters for
high(er)-end
alternatives.

We arrive at five ISE variants using the requirements, the description of 
which is split into
an 
intuitive 
description in the following \SEC[s]
and
a
technical
description
(e.g., a list of instructions and their semantics)
in an associated \APPX.

% =============================================================================


\subsection{Variant 1 (\ISE{1}): \AESFUNC{SubBytes} $+$ \AESFUNC{MixColumn} $+$ explicit \AESFUNC{ShiftRows}}
\label{sec:ise:design:v1}
% =============================================================================

By reproducing~\cite[Section 4.2]{TilGro:06},
\ISE{1}
assumes 
$\RVXLEN = 32$
and adopts a 
column-packed 
representation of state and round key matrices.
As detailed in
\REFFIG{fig:v1:pseudo},
\ISE{1}
adds
$ 4$
instructions ($2$ for encryption, $2$ for decryption).
For example,
\VERB{saes.v1.encs}
applies 
\AESFUNC{SubBytes}  
to elements in   a packed column,
and
\VERB{saes.v1.encm}
applies 
\AESFUNC{MixColumn} 
to               a packed column;
the instruction format for
\VERB{saes.v1.encs}
and
\VERB{saes.v1.encm}
includes $1$ source and $1$ destination register address.
Since 
\VERB{saes.v1.encs}
requires $4$ applications of the S-box, a trade-off between latency and
area is possible st. 
$n$ physical S-box instances are (re)used in $4/n$ cycles
(e.g., $1$ instance in $4$ cycles, or $4$ instances in $1$ cycle).

\REFFIG{fig:v1:round}
demonstrates that use of \ISE{1} to implement AES encryption requires
$47$ instructions per round:
$ 4$ \VERB{lw}           
     instructions to load the round key,
$ 4$ \VERB{xor}           
     instructions to apply \AESFUNC{AddRoundKey},
$ 4$ \VERB{saes.v1.encs}  
     instructions to apply \AESFUNC{SubBytes},
$31$ instructions to apply \AESFUNC{ShiftRows},
and
$ 4$ \VERB{saes.v1.encm}  
     instructions to apply \AESFUNC{MixColumns}.

% =============================================================================

\subsection{Variant 2 (\ISE{2}): \AESFUNC{SubBytes} $+$ \AESFUNC{MixColumn} $+$ implicit \AESFUNC{ShiftRows}}
\label{sec:ise:design:v2}
% =============================================================================

By reproducing~\cite[Section 4.3]{TilGro:06},
\ISE{2}
assumes 
$\RVXLEN = 32$
and adopts a 
column-packed 
representation of state and round key matrices.
As detailed in
\REFFIG{fig:v2:pseudo},
\ISE{2}
adds
$ 4$
instructions ($2$ for encryption, $2$ for decryption).
For example,
\VERB{saes.v2.encs}
applies 
\AESFUNC{SubBytes}  
to elements in   a packed column,
and
\VERB{saes.v2.encm}
applies 
\AESFUNC{MixColumn} 
to               a packed column;
the instruction format for
\VERB{saes.v2.encs}
and
\VERB{saes.v2.encm}
specifies $2$ source and $1$ destination register.
\ISE{2} improves \ISE{1} by applying \AESFUNC{ShiftRows} 
{\em implicitly}:
this is possible by careful indexing of elements in source and destination
columns during application of \AESFUNC{SubBytes} and \AESFUNC{MixColumns},
and also permits
\VERB{saes.v2.encs}
to be used within the key schedule.
The same trade-off is possible as in \ISE{1}, whereby
$n$ physical S-box instances are (re)used in $4/n$ cycles
(e.g., $1$ instance in $4$ cycles, or $4$ instances in $1$ cycle).

\REFFIG{fig:v2:round}
demonstrates that use of \ISE{2} to implement AES encryption requires
$16$ instructions per round:
$ 4$ \VERB{lw}           
     instructions to load the round key,
$ 4$ \VERB{xor}           
     instructions to apply \AESFUNC{AddRoundKey},
$ 4$ \VERB{saes.v1.encs}  
     instructions to apply \AESFUNC{SubBytes},
     and
$ 4$ \VERB{saes.v1.encm}  
     instructions to apply \AESFUNC{MixColumns}.
In the $Nr$-th round, which omits \AESFUNC{MixColumns},
\AESFUNC{ShiftRows} must be applied
{\em explicitly}
using an additional $12$ instructions.

% =============================================================================

\subsection{Variant 3 (\ISE{3}): hardware-assisted T-tables}
\label{sec:ise:design:v3}
% =============================================================================

\ISE{3}
is based on~\cite{NadIkeKur:04,BBFR:06,Saarinen:20}; it
assumes 
$\RVXLEN = 32$
and adopts a 
column-packed 
representation of state and round key matrices.

As detailed in
\REFFIG{fig:v3:pseudo},
\ISE{3}
adds
$ 4$
instructions ($2$ for encryption, $2$ for decryption).
The basic idea is to support an implementation strategy aligned with use
of 
T-tables~\cite[Section 4.2]{DaeRij:02}, 
but compute entries in hardware vs. storing the look-up entries in memory.
For example,
\VERB{saes.v3.encsm}
extracts                     an     element from a packed column,
 applies \AESFUNC{SubBytes}  to the element,
 expands                        the element into a packed column,
 applies \AESFUNC{MixColumn},
then
 applies \AESFUNC{AddRoundKey}.
The inclusion of \AESFUNC{AddRoundKey} follows~\cite{Saarinen:20}, which
improves on~\cite{NadIkeKur:04,BBFR:06}; as a result of this,
the instruction format for
\VERB{saes.v3.encsm}
includes $2$ source and $1$ destination register address.
The requirement for $1$ application of the S-box allows for a more efficient 
functional unit than \ISE{1} or \ISE{2}, for example, either wrt. latency or 
area.

\REFFIG{fig:v3:round}
demonstrates that use of \ISE{3} to implement AES encryption requires
$20$ instructions per round:
$ 4$ \VERB{ lw}           
     instructions to load the round key,
and
$16$ \VERB{saes.v3.encsm} 
     instructions to apply \AESFUNC{SubBytes}, \AESFUNC{ShiftRows}, \AESFUNC{MixColumns}, and \AESFUNC{AddRoundKey}.
In the $Nr$-th round, which omits \AESFUNC{MixColumns},
     \VERB{saes.v3.encsm}
is replaced by 
     \VERB{saes.v3.encs}.

% =============================================================================

\subsection{Variant 4 (\ISE{4}): $64$-bit data-path}
\label{sec:ise:design:v4}
% =============================================================================

\ISE{4} requires $\RVXLEN = 64$
and adopts a {\em double} column-packed 
representation of state and round key matrices,
i.e., {\em two} columns (or $8$ elements) are packed into a $64$-bit word.
It is similar in principle to the SPARC~\cite[Page 109]{SPARC:16} ISE,
but improves on it by adhering to the
$2$ source and $1$ destination register address requirement.
By sourcing two $64$-bit registers, and writing a single $64$-bit register,
a single instruction can 
accept  all  of the current round state as  input
and
produce half of the next    round state as output.

SPARC~\cite[Page 109]{SPARC:16} adds $ 9$
instructions ($4$ for encryption, $4$ for decryption, and $1$ auxiliary).
For example, \VERB{AES_EROUND01} and \VERB{AES_EROUND23}
produce columns $0$ and $1$ and columns $2$ and $3$
respectively.
Each instruction sources $3$ $64$-bit registers, and writes a single
$64$-bit register.

As shown in \REFFIG{fig:v4:pseudo}, \ISE{4} improves this by 
adding only $ 7$
instructions ($2$ for encryption, $2$ for decryption, and $3$ auxiliary).
This is realised by utilising the Equivalent Inverse Cipher representation
detailed in \cite[Section 5.3.5]{FIPS:197}.
This enables all of the round transformations to be applied in the same
order for both encryption and decryption.
The \AESFUNC{AddRoundKey} step can then lifted out of the
round function instructions (where otherwise it would appear in the middle of
the decryption round), and implemented using a base ISA \VERB{xor}
instruction.
The round key then no longer needs to be an input to the instruction,
meaning it only needs $2$ source register operands.

We then note that the nature of \AESFUNC{ShiftRows} means we do
not need separate instructions to compute the next values of
columns (0,1) or columns (2,3) as the SPARC instructions do.
Instead, we can simply reverse the order of the source register
operands, and get the same effect.
This is detailed in \REFFIG{fig:v4:pseudo}, and an example round
function is shown in \REFFIG{fig:v4:round}.

For example,
\VERB{saes.v4.encsm rd, rA, rB}
applies
\AESFUNC{SubBytes}, \AESFUNC{ShiftRow}, and \AESFUNC{MixColumn}  
to elements in a packed column and
produces the {\em next} round values for packed columns (0,1).
Executing
\VERB{saes.v4.encsm rd, rB, rA}, with no change in values of
\VERB{rA} or \VERB{rB}, will produce the next round state values for
packed columns (2, 3).

\REFFIG{fig:v4:round}
demonstrates that use of \ISE{4} to implement AES encryption requires
$ 6$ instructions per round:
$ 2$ \VERB{ld}           
     instructions to load the round key,
$ 2$ \VERB{xor}           
     instructions to apply \AESFUNC{AddRoundKey},
$ 2$ \VERB{saes.v4.encsm}  
     instructions to apply \AESFUNC{SubBytes}, \AESFUNC{ShiftRows}, and \AESFUNC{MixColumns}.
In the $Nr$-th round, which omits \AESFUNC{MixColumns},
     \VERB{saes.v4.encsm}
is replaced by 
     \VERB{saes.v4.encs}.

Note that use of the Equivalent Inverse Cipher representation
necessitates inclusion of the \VERB{saes.v4.imix} instruction, in order
to efficiently imply the inverse \AESFUNC{MixColumn} step to words
of the Key-Schedule.

% =============================================================================

\subsection{Variant 5 (\ISE{5}): quadrant-packed}
\label{sec:ise:design:v5}
% =============================================================================

\begin{figure}[p]
\begin{math}
\begin{tikzpicture}
\matrix [matrix of math nodes,right delimiter={\rbrack},left delimiter={\lbrack}] (S) {
  \AESRND{s}{r}_{0,0} & \AESRND{s}{r}_{0,1} & \AESRND{s}{r}_{0,2} & \AESRND{s}{r}_{0,3} \\
  \AESRND{s}{r}_{1,0} & \AESRND{s}{r}_{1,1} & \AESRND{s}{r}_{1,2} & \AESRND{s}{r}_{1,3} \\
  \AESRND{s}{r}_{2,0} & \AESRND{s}{r}_{2,1} & \AESRND{s}{r}_{2,2} & \AESRND{s}{r}_{2,3} \\
  \AESRND{s}{r}_{3,0} & \AESRND{s}{r}_{3,1} & \AESRND{s}{r}_{3,2} & \AESRND{s}{r}_{3,3} \\
} ;

\matrix at ([xshift={+2.00cm}]  S.east) [matrix of math nodes,right delimiter={\rbrack},left delimiter={\lbrack},anchor={west}] (C0) {
  \AESRND{s}{r}_{0,0} \\ \AESRND{s}{r}_{1,0} \\ \AESRND{s}{r}_{0,1} \\ \AESRND{s}{r}_{1,1} \\
} ;
\matrix at ([xshift={+0.50cm}] C0.east) [matrix of math nodes,right delimiter={\rbrack},left delimiter={\lbrack},anchor={west}] (C1) {
  \AESRND{s}{r}_{0,2} \\ \AESRND{s}{r}_{1,2} \\ \AESRND{s}{r}_{0,3} \\ \AESRND{s}{r}_{1,3} \\
} ;
\matrix at ([xshift={+0.50cm}] C1.east) [matrix of math nodes,right delimiter={\rbrack},left delimiter={\lbrack},anchor={west}] (C2) {
  \AESRND{s}{r}_{2,0} \\ \AESRND{s}{r}_{3,0} \\ \AESRND{s}{r}_{2,1} \\ \AESRND{s}{r}_{3,1} \\
} ;
\matrix at ([xshift={+0.50cm}] C2.east) [matrix of math nodes,right delimiter={\rbrack},left delimiter={\lbrack},anchor={west}] (C3) {
  \AESRND{s}{r}_{2,2} \\ \AESRND{s}{r}_{3,2} \\ \AESRND{s}{r}_{2,3} \\ \AESRND{s}{r}_{3,3} \\
} ;

\node at ($(S.east)!0.5!(C0.west)$) {$\mapsto$} ; \node at ([xshift={-0.50cm}] S.west) [anchor={east}] {$\AESRND{s}{r} = $} ;

\node [inner sep={-2pt},fit=(S-1-1) (S-2-2),  fill={red},   fill opacity={0.2}] {} ;
\node [inner sep={-2pt},fit=(S-1-3) (S-2-4),  fill={green}, fill opacity={0.2}] {} ;
\node [inner sep={-2pt},fit=(S-3-1) (S-4-2),  fill={blue},  fill opacity={0.2}] {} ;
\node [inner sep={-2pt},fit=(S-3-3) (S-4-4),  fill={orange},fill opacity={0.2}] {} ;

\node [inner sep={-2pt},fit=(C0-1-1) (C0-4-1),fill={red},   fill opacity={0.2}] {} ;
\node [inner sep={-2pt},fit=(C1-1-1) (C1-4-1),fill={green}, fill opacity={0.2}] {} ;
\node [inner sep={-2pt},fit=(C2-1-1) (C2-4-1),fill={blue},  fill opacity={0.2}] {} ;
\node [inner sep={-2pt},fit=(C3-1-1) (C3-4-1),fill={orange},fill opacity={0.2}] {} ;
\end{tikzpicture}
\end{math}
\caption{
An illustration of quadrant-packed representation (left), as applied to a state matrix (right).
}
\label{fig:ise:v5:quadpack}
\end{figure}

% -----------------------------------------------------------------------------

\ISE{5}
assumes 
$\RVXLEN = 32$
and adopts a 
novel, {\em quadrant}-packed 
representation of state and round key matrices
as shown in
\REFFIG{fig:ise:v5:quadpack}.
This means that each quadrant of the standard $4\times4$ byte AES state
representation is packed into a single $32$-bit register word.
This allows {\em either} two complete rows (to perform \AESFUNC{ShiftRows}) 
{\em or}
two complete columns (to perform \AESFUNC{MixColumns})
of the state can be accessed by accessing two quadrants.
Based on this, such a representation can
1) afford advantages of {\em both} row- and column-packed alternatives,
   {\em and}
2) allow an instruction format that meets the
   $2$ source and $1$ destination register address constraint of a RISC
   pipeline.
However, it also requires conversion of any input and output 
data between {\em quadrant}-packed and standard {\em column}-packed
representation.
Although such conversion is
amortised by $Nr$ rounds of computation, it still represents an overhead vs.
other variants.

As detailed in \REFFIG{fig:v5:pseudo}, \ISE{5} adds $ 7$
instructions ($3$ for encryption, $3$ for decryption, and $1$ auxiliary).
Taking encryption as an example,
we define two instructions to perform the 
\AESFUNC{ShiftRows} and \AESFUNC{SubBytes} steps.
\VERB{saes.v5.esrsub.lo} performs 
\AESFUNC{ShiftRows} and \AESFUNC{SubBytes} on the two
{\em bottom} quadrants, and \VERB{saes.v5.esrsub.hi} does the same for
the two {\em top} quadrants.
The two instructions are necessary to account for the different rotation
amounts applied to the top and bottom rows as part of \AESFUNC{ShiftRows}.
A single instruction \VERB{saes.v5.emix} applies the \AESFUNC{MixColumns}
transformation to two columns.
The instruction can source two entire column owing to the quadrant
packed representation, but
can only write a single quadrant back.
Hence, two executions of
the same instruction are needed to apply the entire \AESFUNC{MixColumns}
step to each two quadrants.

\REFFIG{fig:v5:round}
demonstrates that use of 
\ISE{5} 
to implement AES encryption requires
$16$ instructions per round:
$ 4$ \VERB{lw}
     instructions to load the round key,
$ 4$ \VERB{xor}
     instructions to apply \AESFUNC{AddRoundKey},
$ 4$ \VERB{saes.v5.esrsub.[lo|hi]}
     instructions to apply \AESFUNC{SubBytes} and \AESFUNC{ShiftRows},
     and
$ 4$ \VERB{saes.v5.emix}
     instructions to apply \AESFUNC{MixColumns}.
Note that conversion into (resp. from) quadrant-packed representation
requires a further
$12$ instructions;
     this can be reduced to
$ 4$ \VERB{pack[h]}
     instructions using the 
     standard 
     bit-manipulation
     extension~\cite[Section 17]{RV:ISA:I:19}.

\ISE{5} instructions may be implemented with between $1$ and $4$
SBox instances, with a corresponding tradeoff between area and
latency.
As with \ISE{1} and \ISE{2} however, additional storage elements
are required if fewer than $4$ SBoxes are instanced in order to
store intermediate results.
The auxiliary \VERB{saes.v5.sub} instruction is used during the
Key-Schedule, and can act simply as an interface to the SBoxes
already required by the round instructions.

% =============================================================================



%
\begin{figure}[h!]
\centering
\begin{subfigure}[b]{0.45\textwidth}
\includegraphics[width={\textwidth}]{diagrams/ise-datapath-v1.png}
\caption{
  A diagrammatic description of the functional unit required to support \ISE{1}.
}
\label{fig:v1:fu}
\end{subfigure}
\begin{subfigure}[b]{0.45\textwidth}
\includegraphics[width={\textwidth}]{diagrams/ise-datapath-v2.png}
\caption{
  A diagrammatic description of the functional unit required to support \ISE{2}.
}
\label{fig:v2:fu}
\end{subfigure}

\begin{subfigure}[b]{0.45\textwidth}
\includegraphics[width={\textwidth}]{diagrams/ise-datapath-v3.png}
\caption{
  A diagrammatic description of the functional unit required to support \ISE{3}.
}
\label{fig:v3:fu}
\end{subfigure}
\begin{subfigure}[b]{0.45\textwidth}
\includegraphics[width={\textwidth}]{diagrams/ise-datapath-v4.png}
\caption{
  A diagrammatic description of the functional unit required to support \ISE{4}.
}
\label{fig:v4:fu}
\end{subfigure}

\begin{subfigure}[b]{0.45\textwidth}
\includegraphics[width={\textwidth}]{diagrams/ise-datapath-v5.png}
\caption{
  A diagrammatic description of the functional unit required to support \ISE{5}.
}
\label{fig:v5:fu}
\end{subfigure}
\end{figure}


\begin{figure}[!h]
\begin{lstlisting}[language=pseudo,style=block]
saes.v1.encs rd, rs1 : v1.SubBytes(rd, rs1, fwd=1)
saes.v1.decs rd, rs1 : v1.SubBytes(rd, rs1, fwd=0)
saes.v1.encm rd, rs1 : v1.MixColumn(rd, rs1, fwd=1)
saes.v1.decm rd, rs1 : v1.MixColumn(rd, rs1, fwd=0)

v1.SubByte(rd, rs1, fwd):
    rd.8[i] = AESSBox[rs1.8[i]] if fwd else AESInbSBox[rs1.8[i]] for i=0..3

v1.MixColumn(rd, rs1, fwd):
    for i=0..3:
        tmp.32  = ROTL32(rs1.32, 8*i)
        rd.8[i] = AESMixColumn(tmp.32) if fwd else AESInvMixColumn(tmp.32)
\end{lstlisting}
\caption{
  Instruction mnemonics, and their mapping onto pseudo-code functions, for \ISE{1}.
}
\label{fig:v1:pseudo}
\end{figure}

\begin{figure}[!h]
\begin{lstlisting}[language=pseudo,style=block]
saes.v2.encs rd, rs1, rs2 : v2.SubBytes(rd, rs1, rs2, fwd=1)
saes.v2.decs rd, rs1, rs2 : v2.SubBytes(rd, rs1, rs2, fwd=0)
saes.v2.encm rd, rs1, rs2 : v2.MixColumns(rd, rs1, rs2, fwd=1)
saes.v2.decm rd, rs1, rs2 : v2.MixColumns(rd, rs1, rs2, fwd=0)

v2.SubBytes(rd, rs1, rs2, fwd):
  t1.32  = {rs1.8[0], rs2.8[1], rs1.8[2], rs2.8[3]}
  rd.8[i]= AESSBox[t1.8[i]] if fwd else AESInvSBox[t1.8[i]] for i=0..3

v2.MixColumns(rd, rs1, rs2, fwd):
  t1.32  = {rs1.8[0], rs1.8[1], rs2.8[2], rs2.8[3]}
  for i=0..3:
      tmp.32 = ROTL32(rs1.32, 8*i)
      rd.8[i]= AESMixColumn(tmp.32) if fwd else AESInvMixColumn(tmp.32)
\end{lstlisting}
\caption{
  Instruction mnemonics, and their mapping onto pseudo-code functions, for \ISE{2}.
}
\label{fig:v2:pseudo}
\end{figure}

\begin{figure}[!h]
\begin{lstlisting}[language=pseudo,style=block]
saes.v3.encs  rd, rs1, rs2, bs : v3.Proc(rd, rs1, rs2, bs, fwd=1, mix=0)
saes.v3.encsm rd, rs1, rs2, bs : v3.Proc(rd, rs1, rs2, bs, fwd=1, mix=1)
saes.v3.decs  rd, rs1, rs2, bs : v3.Proc(rd, rs1, rs2, bs, fwd=0, mix=0)
saes.v3.decsm rd, rs1, rs2, bs : v3.Proc(rd, rs1, rs2, bs, fwd=0, mix=1)

v3.Proc(rd, rs1, rs2, bs, fwd, mix):
  x     = AESSBox[rs2.8[bs]] if fwd else AESInvSBox[rs2.8[bs]]
  if   mix and  fwd: t1.32 = {GFMUL(x, 3),      x    ,      x   ,GFMUL(x, 2)}
  elif mix and !fwd: t1.32 = {GFMUL(x,11),GFMUL(x,13),GFMUL(x,9),GFMUL(x,14)}
  else             : t1.32 = {0, 0, 0, x}
  rd.32 = ROTL32(t1.32, 8*bs) ^ rs1
\end{lstlisting}
\caption{
  Instruction mnemonics, and their mapping onto pseudo-code functions, for \ISE{3}.
}
\label{fig:v3:pseudo}
\end{figure}

\begin{figure}[!h]
\begin{lstlisting}[language=pseudo,style=block]
saes.v4.ks1       rd rs1 rcon : v4.ks1(rd, rs1, rcon)
saes.v4.ks2       rd rs1 rs2  : v4.ks2(rd, rs1, rs2 )
saes.v4.imix      rd rs1      : v4.InvMix(rd, rs1)
saes.v4.encsm     rd rs1 rs2  : v4.Enc(rd, rs1, rs2, mix=1)
saes.v4.encs      rd rs1 rs2  : v4.Enc(rd, rs1, rs2, mix=0)
saes.v4.decsm     rd rs1 rs2  : v4.Dec(rd, rs1, rs2, mix=1)
saes.v4.decs      rd rs1 rs2  : v4.Dec(rd, rs1, rs2, mix=0)

v4.ks1(rd, rs1, enc_rcon):     // KeySchedule: SubBytes, Rotate, Round Const
    temp.32   = rs1.32[1]
    rcon      = 0x0
    if(enc_rcon != 0xA):
        temp.32 = ROTR32(temp.32, 8)
        rcon    = RoundConstants.8[enc_rcon]
    temp.8[i] = AESSBox[temp.8[i]]  for i=0..3
    temp.8[0] = temp.8[0] ^ rcon
    rd.64     = {temp.32, temp.32}

v4.ks2(rd, rs1, rs2):           // KeySchedule: XOR
    rd.32[0]  = rs1.32[1] ^ rs2.32[0]
    rd.32[1]  = rs1.32[1] ^ rs2.32[0] ^ rs2.32[1]

v4.Enc(rd, rs1, rs2, mix): // SubBytes, ShiftRows, MixColumns
    t1.128    = ShiftRows({rs2, rs1})
    t2.64     = t1.64[0]
    t3.8[i]   = AESSBox[t2.8[i]] for i=0..7
    rd.32[i]  = AESMixColumn(t3.32[i]) if mix else t3.32[i] for i=0..1

v4.Dec(rd, rs1, rs2, mix, hi): // InvSubBytes, InvShiftRows, InvMixColumns
    t1.128    = InvShiftRows(rs2 || rs1)
    t2.64     = t1.64[0]
    t3.8[i]   = AESInvSBox[t2.8[i]] for i=0..7
    rd.32[i]  = AESInvMixColumn(t3.32[i]) if mix else t3.32[i] for i=0..1

v4.InvMix(rd, rs1):             // Inverse MixColumns
    rd.32[i]  = AESInvMixColumn(rs1.32[i]) for i=0..1
\end{lstlisting}
\caption{
  Instruction mnemonics, and their mapping onto pseudo-code functions, for \ISE{4}.
}
\label{fig:v4:pseudo}
\end{figure}

\begin{figure}[!h]
\begin{lstlisting}[language=pseudo,style=block]
saes.v5.esrsub.lo rd, rs1, rs2 : rd = v5.SrSub(rs1, rs2, fwd=1, hi=0)
saes.v5.esrsub.hi rd, rs1, rs2 : rd = v5.SrSub(rs1, rs2, fwd=1, hi=1)
saes.v5.dsrsub.lo rd, rs1, rs2 : rd = v5.SrSub(rs1, rs2, fwd=0, hi=0)
saes.v5.dsrsub.hi rd, rs1, rs2 : rd = v5.SrSub(rs1, rs2, fwd=0, hi=1)
saes.v5.emix      rd, rs1, rs2 : rd = v5.Mix(rs1, rs2, fwd=1)
saes.v5.dmix      rd, rs1, rs2 : rd = v5.Mix(rs1, rs2, fwd=0)
saes.v5.sub       rd, rs1      : rd = SubBytes(rs1.8[i])         for i=0..3

v5.SrSub(rd, rs1, rs2, fwd, hi):
  if(fwd):
    if hi: tmp.32 = {rs1.8[3], rs2.8[0], rs2.8[1], rs2.8[2]}
    else : tmp.32 = {rs2.8[3], rs1.8[1], rs1.8[0], rs1.8[2]}
    tmp.8[i]      =    AESSBox[tmp.8[i]] for i=0..3
  else:
    if hi: tmp.32 = {rs2.8[3], rs2.8[0], rs1.8[1], rs2.8[2]}
    else : tmp.32 = {rs1.8[3], rs2.8[1], rs1.8[0], rs1.8[2]}
    tmp.8[i]      = InvAESSBox[tmp.8[i]] for i=0..3
  if(hi): rd.32 = {tmp.8[2],tmp.8[3],tmp.8[0],tmp.8[1]}
  else  : rd.32 = {tmp.8[1],tmp.8[3],tmp.8[0],tmp.8[2]}

v5.mix(rd, rs1, rs2, fwd):
  col0.32 = {rs1.8[2], rs1.8[3], rs2.8[2], rs2.8[3]}
  col1.32 = {rs1.8[0], rs1.8[1], rs2.8[0], rs2.8[1]}
  n0.8    = AESMixColumn(       col0   ) if fwd else AESInvMixColumn(       col0   )
  n1.8    = AESMixColumn(ROTL32(col0,8)) if fwd else AESInvMixColumn(ROTL32(col0,8))
  n2.8    = AESMixColumn(       col1   ) if fwd else AESInvMixColumn(       col1   )
  n3.8    = AESMixColumn(ROTL32(col1,8)) if fwd else AESInvMixColumn(ROTL32(col1,8))
  rd.32 = {n2, n3, n0, n1}
\end{lstlisting}
\caption{
  Instruction mnemonics, and their mapping onto pseudo-code functions, for \ISE{5}.
}
\label{fig:v5:pseudo}
\end{figure}


% -----------------------------------------------------------------------------

\subsection{Implementation}
\label{sec:ise:imp}
% =============================================================================

The evaluation of each ISE considers two different RISC-V compliant base
micro-architectures, which constitute two different host cores:

\begin{itemize}
\item The \CORE{2}\footnote{%
        \ifbool{anonymous}{Details of this core have been anonymised to comply with the TCHES submission guidelines.}{\url{https://github.com/scarv/scarv}}
      } core 
      supports the 
      RV32IMC 
      instruction set, i.e.,
      the 
             $32$-bit~\cite[Section 2]{RV:ISA:I:19} 
      base integer ISA plus 
      standard 
      Multiplication ~\cite[Section  7]{RV:ISA:I:19}
      and
      Compressed ~\cite[Section 16]{RV:ISA:I:19}
      extensions.
      Per the block diagram shown in~\REFFIG{fig:core:2:normal},
      the core 
      executes instructions using a $5$-stage, in-order pipeline.
      No branch prediction is supported.
      There are two memory interfaces for instruction fetch and data memory
      accesses.
      No instruction or data caches are supported.
      The core implements various performance counters,
      and
      elements of the
      RISC-V Privileged Resource Architecture (PRA)~\cite[Chapter 3]{RV:ISA:II:19}
      related to exception and interrupt handling.

\item The \CORE{1}~\cite{rocket:16} 
        core
      executes instructions using a $5$-stage, in-order pipeline
      which is highly configurable.
      We take advantage of this, considering two variants whose
      exact configuration is outlined in
      \REFFIG{fig:rocket:32} 
      and 
      \REFFIG{fig:rocket:64}:
      the variants represent single $32$-bit and $64$-bit cores respectively,
      and so
      support  the 
      RV32IMC 
      (resp. RV64IMC)
      instruction set, i.e.,
      the 
             $32$-bit~\cite[Section 2]{RV:ISA:I:19} 
      (resp. $64$-bit~\cite[Section 5]{RV:ISA:I:19})
      base integer ISA plus 
      standard 
      Multiplication ~\cite[Section  7]{RV:ISA:I:19}
      and
      Compressed ~\cite[Section 16]{RV:ISA:I:19}
      extensions.
      Each variant is configured to support
      an instruction cache, 
      a  data        cache,
      and
      a  branch prediction mechanism,
      but 
      no floating-point support.

\end{itemize}

\noindent
To support each ISE, two modifications were made to each host core:
the instruction decoder was modified to support
operand selection
and
an AES Functional Unit (AES-FU) was added to support execution of
ISE instructions.
The \CORE{2} core integrates the AES-FU directly into the pipeline,
while
the \CORE{1} core accesses the AES-FU via the
Rocket Custom Coprocessor (RoCC)~\cite[Section 4]{rocket:16}
interface.
Since \REFREQ{req:2}
(each instruction uses at most $2$ source and $1$ destination register)
is fulfilled,
neither micro-architecture required further structural alteration.
A synthesis-time parameter was used to switch between different 
ISEs.

% =============================================================================


% -----------------------------------------------------------------------------

\subsection{Evaluation}

\paragraph{Hardware.}
\label{sec:ise:eval:hw}
% =============================================================================

Each ISE variant was integrated into the two host cores 
described in \REFSEC{sec:ise:imp}.
The variants which assume  $\RVXLEN = 32$
(\ISE{1}, \ISE{2}, \ISE{3}, and \ISE{5}) 
were evaluated
on {\em both} the
$32$-bit \CORE{2} core
{\em  and} the
$32$-bit \CORE{1} core;
the variant  which assumes $\RVXLEN = 64$
(\ISE{4})
was  evaluated
on {\em only} the
$64$-bit \CORE{1} core.
For \ISE{1}, \ISE{2} and \ISE{5} a trade-off
between latency and area exists. 
Each such case is considered through two optimisation goals:
the (A)rea    goal
instantiates $1$ S-box   and has a $n$-cycle execution latency,
whereas
the (L)atency goal
instantiates $4$ S-boxes and has a $1$-cycle execution latency.

\REFTAB{tab:eval:hw:encdec}
shows the metrics associated with the hardware implementations, 
and show the separated cost of the standalone ISE logic, and the
cost of the core and integrated ISE.
We use the open source Yosys~\cite{yosys} synthesis tool (v0.9+1706)
with default settings
to provide post synthesis circuit area (in the form of NAND2 gate
equivalents) and circuit depths (in the form of gate delays).
This approach, while more abstract than providing exact area and
frequency results for a particular ASIC standard cell library, is
much easier to reproduce\footnote{
Especially
for researchers lacking expensive commercial
synthesis tools and process design kits.
} while still providing meaningful results.
We focus on ASIC implementations (rather than FPGA implementations)
because this is the more relevant metric to the industrial (rather than
academic) RISC-V community.
This methodology has also been used for other RISC-V standard extension
proposals, namely the bit-manipulation extension~\cite[Section 3.1, Page 54]{riscv:bitmanip:draft}.
We found 
that none of the ISEs affected the critical path of either the \CORE{2} 
or \CORE{1} core.
Considering each ISE as implemented on the \CORE{1} core, we note the 
overhead wrt. area is marginal: this stems from the fact that the 
baseline area of \CORE{1} includes the data and instruction caches.

In
\REFTAB{tab:eval:hw:dec}
we consider the hardware costs when only {\em encryption} instructions are
implemented.
This is relevant to systems which only care about certain block cipher
modes of operation, such as Galos/Counter-mode,
which only use the encryption function of a block cipher.
We discuss this further in \REFSEC{sec:gcm}.

% =============================================================================


\paragraph{Software.}
\label{sec:ise:eval:sw}
\input{tex/body-ise_eval_sw.tex}

\paragraph{Discussion.}
\label{sec:ise:eval:discuss}
% =============================================================================

\REFTAB{tab:eval:hw}
demonstrates that all ISE variants
imply a modest area overhead relative to their host core.
For the RV32 \CORE{1} the area overhead of a synthesised \CORE{1} Tile with
caches was less than $1\%$ in all cases.
For the \CORE{2}, the area overhead ranged between
$13\%$ (\ISE{5} (L))
and
$3\%$ (\ISE{3}).
\REFTAB{tab:eval:sw:size}
shows all ISE variants
having similarly small memory footprints in terms of both instruction code and
data.
Beyond this, and per 
\REFSEC{sec:ise:design},
the primary metric of interest is efficiency in terms of
the latency-area product.
this metric draws on data from
\REFTAB{tab:eval:hw}
plus either
\REFTAB{tab:eval:sw:perf:2}
or
\REFTAB{tab:eval:sw:perf:1}
for the \CORE{2} or \CORE{1} core respectively.
We deliberately omit the area of the host core from this calculation, as this
fixed overhead dominates the final value and detracts from the comparison
between ISEs themselves.

\REFTAB{tab:eval:results} 
captures the results for the \CORE{1} core, although the same conclusion can 
be drawn for the \CORE{2} core.  Qualitatively, we place more of a weight on 
Encryption (\ALG{Enc})
and 
Decryption (\ALG{Dec})
vs.
Encryption Key Expansion (\ALG{Enc-KeyExp})
and 
Decryption Key Expansion (\ALG{Dec-KeyExp}),
because
typically many \ALG{Enc} or \ALG{Dec} operations are performed per
\ALG{KeyExp}.
For a $32$-bit core, our conclusion is that
\ISE{3} 
is the best option.
Despite not being the fastest (by a small margin), it is the most efficient,
and simplest to implement.
The area optimised \ISE{2} implementation sometimes comes close in
efficiency, but requires a more complex multi-cycle implementation
in this case.
For a $64$-bit core,
\ISE{4} 
is the best option, which is somewhat obvious because it specifically makes
use of the wider data-path.
With reference to
\REFTAB{tab:eval:sw:perf:1}, 
note that the number of cycles per instruction executed is relatively low.
This fact stems from use of the ROCC interface, in that forwarding of the 
result from an ISE instruction (that uses the ROCC) incurs an overhead vs. 
an ISE instruction; fine-grained integration of the AES-FU could therefore
incrementally improve the results.

We believe it is sensible to standardise different ISEs for the
RV32 and RV64 base ISAs.
This allows each ISE design to better suit the constraints of each
base ISA.
In the RV32 case, this acknowledges that such cores will most often
appear in resource-constrained, embedded or IoT class devices.
Hence, the most efficient ISE design is appropriate.
For necessarily larger RV64-based designs, it makes sense to take advantage
of the wider data-path, and acknowledge that these are more likely to
be application class cores. Hence, they will place a higher value
on performance than area-efficiency.

% -----------------------------------------------------------------------------

\begin{adjustbox}{center,caption={
    Hardware implementation metrics for each ISE variant with
    encrypt and decrypt instructions implemented.
                                 },label={tab:eval:hw:encdec},float={table}[!p]}
\centering
\begin{tabular}{|c|l|rr|r|r|}
\hline
  \multicolumn{1}{|c|}{ISA}
& \multicolumn{1}{ c|}{Variant}
& \multicolumn{1}{ c|}{             ISE}
& \multicolumn{1}{ c|}{       ISE Path }
& \multicolumn{1}{ c|}{\CORE{2}     CPU}
& \multicolumn{1}{ c|}{\CORE{1}     CPU}
\\
& \multicolumn{1}{ c|}{/ Goal       }
& \multicolumn{1}{ c|}{Area         }
& \multicolumn{1}{ c|}{Depth        }
& \multicolumn{1}{ c|}{$+$ ISE area }
& \multicolumn{1}{ c|}{$+$ ISE area }
\\
\hline
\hline
 RV32IMC & Baseline    &              &            &       37325  ($1.00\times$) &       3501576 ($1.000\times$) \\
 RV32IMC & \ISE{1} (L) &        3514  & \bftab 18  &       41746  ($1.  \times$) &       3508448 ($1.002\times$) \\
 RV32IMC & \ISE{1} (A) &        2195  &        21  &       40171  ($1.  \times$) &       3506995 ($1.002\times$) \\
 RV32IMC & \ISE{2} (L) &        3574  &        19  &       41132  ($1.  \times$) &       3508946 ($1.002\times$) \\
 RV32IMC & \ISE{2} (A) &        1355  &        21  &       38777  ($1.  \times$) &\bftab 3506591 ($1.001\times$) \\
 RV32IMC & \ISE{3}     & \bftab 1149  &        30  &\bftab 38546  ($1.  \times$) &       3506761 ($1.001\times$) \\
 RV32IMC & \ISE{5} (L) &        4172  &        21  &       42035  ($1.  \times$) &       3510055 ($1.002\times$) \\
 RV32IMC & \ISE{5} (A) &        1726  &        23  &       39144  ($1.  \times$) &       3507755 ($1.002\times$) \\
\hline
\hline
 RV64IMC & Baseline &          &          &  N/A  & 3717607 (1.000$\times$) \\
 RV64IMC & \ISE{4}  &     8226 &       28 &  N/A  & 3733786 (1.004$\times$) \\
\hline
\end{tabular}
\end{adjustbox}


\begin{adjustbox}{center,caption={
    Hardware implementation metrics for each ISE variant with
    only encrypt instructions implemented.
                                 },label={tab:eval:hw:dec},float={table}[!p]}
\centering
\begin{tabular}{|c|l|rr|r|r|}
\hline
  \multicolumn{1}{|c|}{ISA}
& \multicolumn{1}{ c|}{Variant}
& \multicolumn{1}{ c|}{             ISE}
& \multicolumn{1}{ c|}{       ISE Path }
& \multicolumn{1}{ c|}{\CORE{2}     CPU}
& \multicolumn{1}{ c|}{\CORE{1}     CPU}
\\
& \multicolumn{1}{ c|}{/ Goal       }
& \multicolumn{1}{ c|}{Area         }
& \multicolumn{1}{ c|}{Depth        }
& \multicolumn{1}{ c|}{$+$ ISE area }
& \multicolumn{1}{ c|}{$+$ ISE area }
\\
\hline
\hline
 RV32IMC & Baseline    &              &            &       37325  ($1.00\times$) &       3501576 ($1.000\times$) \\
 RV32IMC & \ISE{1} (L) &        1605  & \bftab 17  &       39154  ($    \times$) &       3506224 ($     \times$) \\
 RV32IMC & \ISE{1} (A) &        1038  &        23  &       38561  ($    \times$) &       3505695 ($     \times$) \\
 RV32IMC & \ISE{2} (L) &        1611  & \bftab 17  &       40337  ($    \times$) &       3506729 ($     \times$) \\
 RV32IMC & \ISE{2} (A) &         780  &        21  &       38479  ($    \times$) &       3505910 ($     \times$) \\
 RV32IMC & \ISE{3}     & \bftab  630  &        25  &       38301  ($    \times$) &       3506097 ($     \times$) \\
 RV32IMC & \ISE{5} (L) &        1852  &        23  &       40626  ($    \times$) &       3507518 ($     \times$) \\
 RV32IMC & \ISE{5} (A) &        1048  &        23  &       38749  ($    \times$) &       3506816 ($     \times$) \\
\hline
\hline
 RV64IMC & Baseline &          &          &  N/A  & 3717607 (1.000$\times$) \\
 RV64IMC & \ISE{4}  &  3790    &    27    &  N/A  &         (     $\times$) \\
\hline
\end{tabular}
\end{adjustbox}



\begin{adjustbox}{center,caption={Software  memory footprint measured in bytes
                                  for each ISE variant.
                                 },label={tab:eval:sw:size},float={table}[!p]}
\centering
\begin{tabular}{|c|c|r|r|r|r|r|}
\hline
  \multicolumn{1}{|c|}{ISA}
& \multicolumn{1}{ c|}{Variant}
& \multicolumn{1}{ c|}{$\ALG{Enc}$}
& \multicolumn{1}{ c|}{$\ALG{Dec}$}
& \multicolumn{1}{ c|}{$\ALG{Enc-KeyExp}$}
& \multicolumn{1}{ c|}{$\ALG{Dec-KeyExp}$}
& \multicolumn{1}{ c|}{.data} 
\\
\hline
\hline
%RV32IMC & Byte    &            &           &      312 &        0 &  522 \\
 RV32IMC & T-table &       804  &       804 &      154 &      174 & 5120 \\
 RV32IMC & \ISE{1} &       424  &       424 &\bftab 68 &        0 &   10 \\
 RV32IMC & \ISE{2} &\bftab 234  &\bftab 238 &\bftab 68 &       62 &   10 \\
 RV32IMC & \ISE{3} &       290  &       290 &       86 &       64 &   10 \\
 RV32IMC & \ISE{5} &       266  &       278 &      290 &        0 &   10 \\
\hline
 RV64IMC & \ISE{4} &       268  &       268 &      168 &      100 &    0 \\
\hline
\end{tabular}
\end{adjustbox}

\begin{adjustbox}{center,caption={Execution metrics
                                  for each ISE variant on the \CORE{2} core.
                                  Note that the $64$-bit \ISE{4} is absent, since there is no $64$-bit \CORE{2} core.
                                 },label={tab:eval:sw:perf:2},float={table}[!p]}
\centering
\begin{tabular}{|c|l|rr|rr|rr|rr|}
\hline
  \multicolumn{1}{|c|}{ISA}
& \multicolumn{1}{ c|}{Variant}
& \multicolumn{2}{ c|}{$\ALG{Enc}$}
& \multicolumn{2}{ c|}{$\ALG{Dec}$}
& \multicolumn{2}{ c|}{$\ALG{Enc-KeyExp}$}
& \multicolumn{2}{ c|}{$\ALG{Dec-KeyExp}$}
\\
\cline{3-10}
& / Goal
& \multicolumn{1}{ c|}{iret}
& \multicolumn{1}{ c|}{cycles}
& \multicolumn{1}{ c|}{iret}
& \multicolumn{1}{ c|}{cycles}
& \multicolumn{1}{ c|}{iret}
& \multicolumn{1}{ c|}{cycles}
& \multicolumn{1}{ c|}{iret}
& \multicolumn{1}{ c|}{cycles}
\\
\hline
\hline
%RV32IMC & Byte        &            &            &            &            &            &            &            &            \\
 RV32IMC & T-table     &          938 &         1016 &          938 &         1037&          430 &          515 &         1711 &         2307 \\ 
 RV32IMC & \ISE{1} (L) &          512 &          575 &          512 &          576& \bftab   198 & \bftab   302 & \bftab   204 & \bftab   321 \\
 RV32IMC & \ISE{1} (A) &          512 &          735 &          512 &          736& \bftab   198 &          342 & \bftab   204 &          361 \\
 RV32IMC & \ISE{2} (L) & \bftab   215 & \bftab   274 & \bftab   216 & \bftab   285& \bftab   198 & \bftab   302 &          335 &          615 \\
 RV32IMC & \ISE{2} (A) & \bftab   215 &          501 & \bftab   216 &          522& \bftab   198 &          332 &          335 &          753 \\
 RV32IMC & \ISE{3}     &          238 &          291 &          238 &          286&          219 &          312 &          659 &         1118 \\
 RV32IMC & \ISE{5} (L) &          227 &          294 &          227 &          291&          332 &          449 &          338 &          468 \\
 RV32IMC & \ISE{5} (A) &          227 &          554 &          227 &          532&          332 &          479 &          338 &          498 \\
\hline
\end{tabular}                
\end{adjustbox}

\begin{adjustbox}{center,caption={Execution metrics
                                  for each ISE variant on the \CORE{1} core.
                                  Note that the $64$-bit \ISE{4} uses the $64$-bit \CORE{1} core; all others use the $32$-bit \CORE{1} core.
                                 },label={tab:eval:sw:perf:1},float={table}[!p]}
\centering
\begin{tabular}{|c|l|rr|rr|rr|rr|}
\hline
  \multicolumn{1}{|c|}{ISA}
& \multicolumn{1}{ c|}{Variant}
& \multicolumn{2}{ c|}{$\ALG{Enc}$}
& \multicolumn{2}{ c|}{$\ALG{Dec}$}
& \multicolumn{2}{ c|}{$\ALG{Enc-KeyExp}$}
& \multicolumn{2}{ c|}{$\ALG{Dec-KeyExp}$}
\\
\cline{3-10}
& / Goal
& \multicolumn{1}{ c|}{iret}
& \multicolumn{1}{ c|}{cycles}
& \multicolumn{1}{ c|}{iret}
& \multicolumn{1}{ c|}{cycles}
& \multicolumn{1}{ c|}{iret}
& \multicolumn{1}{ c|}{cycles}
& \multicolumn{1}{ c|}{iret}
& \multicolumn{1}{ c|}{cycles}
\\
\hline
\hline
%RV32IMC & Byte        &            &            &            &            &            &            &            &            \\
 RV32IMC & T-table     &       934  &      1338  &       934  &      1003  &       430  &       569  &      1711  &      2167  \\
 RV32IMC & \ISE{1} (L) &       513  &       659  &       513  &       613  &\bftab 199  &       268  &\bftab 200  &\bftab 270  \\
 RV32IMC & \ISE{1} (A) &       513  &       791  &       513  &       725  &\bftab 199  &       308  &\bftab 200  &       310  \\
 RV32IMC & \ISE{2} (L) &\bftab 216  &\bftab 351  &\bftab 217  &       354  &\bftab 199  &\bftab 263  &       336  &       496  \\
 RV32IMC & \ISE{2} (A) &\bftab 216  &       503  &\bftab 217  &       534  &\bftab 199  &       293  &       336  &       637  \\
 RV32IMC & \ISE{3}     &       239  &       396  &       239  &       410  &       220  &       462  &       660  &      2420  \\
 RV32IMC & \ISE{5} (L) &       228  &       366  &       228  &\bftab 322  &       333  &       405  &       334  &       404  \\
 RV32IMC & \ISE{5} (A) &       228  &       536  &       228  &       546  &       333  &       438  &       334  &       434  \\
\hline
 RV64IMC & T-table     &       934  &      1086  &       934  &      1015  &       431  &       479  &      1712  &      1995  \\
 RV64IMC & \ISE{4}     &        76  &       115  &        76  &       133  &        61  &       199  &       131  &       286  \\
\hline
\end{tabular}
\end{adjustbox}

% -----------------------------------------------------------------------------

\begin{adjustbox}{center,caption={
    Comparison of performance/area product. 
},label={tab:eval:results},float={table}[!t]}
\centering
\begin{tabular}{|c|l|rr|rr|rr|rr|}
\hline
  \multicolumn{1}{|c|}{ISA}
& \multicolumn{1}{ c|}{Variant}
& \multicolumn{2}{ c|}{$\ALG{Enc}$}
& \multicolumn{2}{ c|}{$\ALG{Dec}$}
& \multicolumn{2}{ c|}{$\ALG{EncKeyExp}$}
& \multicolumn{2}{ c|}{$\ALG{DecKeyExp}$}
\\
\cline{3-10}
& / Goal
& \multicolumn{1}{ c|}{iret}
& \multicolumn{1}{ c|}{cycles}
& \multicolumn{1}{ c|}{iret}
& \multicolumn{1}{ c|}{cycles}
& \multicolumn{1}{ c|}{iret}
& \multicolumn{1}{ c|}{cycles}
& \multicolumn{1}{ c|}{iret}
& \multicolumn{1}{ c|}{cycles}
\\
\hline
\hline
RV32IMC & \ISE{1} (L) &       0.748 &       0.840 &       0.748 &       0.841 &       0.289 &       0.441 &       0.298 &       0.469 \\
RV32IMC & \ISE{1} (A) &       0.468 &       0.672 &       0.468 &       0.673 &       0.181 &       0.313 &\bftab 0.187 &\bftab 0.330 \\
RV32IMC & \ISE{2} (L) &       0.321 &       0.409 &       0.322 &       0.425 &       0.295 &       0.451 &       0.500 &       0.918 \\
RV32IMC & \ISE{2} (A) &       0.125 &       0.291 &       0.125 &       0.303 &       0.115 &       0.193 &       0.195 &       0.437 \\
RV32IMC & \ISE{3}     &\bftab 0.116 &\bftab 0.142 &\bftab 0.116 &\bftab 0.139 &\bftab 0.107 &\bftab 0.152 &       0.321 &       0.544 \\
RV32IMC & \ISE{5} (L) &       0.394 &       0.510 &       0.394 &       0.504 &       0.576 &       0.778 &       0.586 &       0.811 \\
RV32IMC & \ISE{5} (A) &       0.184 &       0.449 &       0.184 &       0.431 &       0.269 &       0.388 &       0.274 &       0.404 \\
\hline
RV64IMC & \ISE{4}     &       0.266 &       0.402 &       0.266 &       0.465 &       0.213 &       0.696 &       0.458 &       1.000  \\
\hline
\end{tabular}
\end{adjustbox}

% =============================================================================


% =============================================================================

\section{Using ISEs to implement AES-GCM}
\label{sec:gcm}
%\input{tex/body-modes.tex}
% =============================================================================

\begin{table}[p]
\centering
\begin{tabular}{|c|c|c|rrrrrr|}
\hline
ISA    & Karatsuba & Reduce & \VERB{grev}
                            & \VERB{xor}
                            & \VERB{s[lr]li}
                            & \VERB{clmul} 
                            & \VERB{clmulh}
                            & Total \\
\hline
\hline
RV32IB &        no &    mul &$  4$&$ 36$&$  0$&$ 20$&$ 20$&$ 80$ \\
RV32IB &        no &  shift &$  4$&$ 56$&$ 24$&$ 16$&$ 16$&$116$ \\
RV32IB &       yes &    mul &$  4$&$ 52$&$  0$&$ 13$&$ 13$&$ 82$ \\
RV32IB &       yes &  shift &$  4$&$ 72$&$ 24$&$  9$&$  9$&$118$ \\
\hline
RV64IB &        no &    mul &$  2$&$ 10$&$  0$&$  6$&$  6$&$ 24$ \\
RV64IB &        no &  shift &$  2$&$ 20$&$ 12$&$  4$&$  4$&$ 42$ \\
RV64IB &       yes &    mul &$  2$&$ 14$&$  0$&$  5$&$  5$&$ 26$ \\
RV64IB &       yes &  shift &$  2$&$ 24$&$ 12$&$  3$&$  3$&$ 44$ \\
\hline
\end{tabular}
\caption{Instruction counts for multiplication in $\F_{2^{128}}$ as used by \ALG{GHASH}.}
\label{tab:gcm:instrs}
\end{table}

\begin{table}[p]
\centering
\begin{tabular}{|c|c|c|rrrr|}
\hline
ISA    & Karatsuba & Reduce & $1$-cycle       & $2$-cycle       & $3$-cycle       & $6$-cycle       \\
       &           &        & \VERB{clmul[h]} & \VERB{clmul[h]} & \VERB{clmul[h]} & \VERB{clmul[h]} \\
\hline
\hline
RV32IB &        no &    mul &     \bftab  80  &            120  &            160  &            280  \\
RV32IB &        no &  shift &            116  &            148  &            180  &            276  \\
RV32IB &       yes &    mul &             82  &    \bftab  108  &     \bftab 134  &            212  \\
RV32IB &       yes &  shift &            118  &            136  &            154  &     \bftab 208  \\
\hline
RV64IB &        no &    mul &     \bftab  24  &    \bftab   36  &             48  &             84  \\
RV64IB &        no &  shift &             42  &             50  &             58  &             82  \\
RV64IB &       yes &    mul &             26  &    \bftab   36  &     \bftab  46  &             76  \\
RV64IB &       yes &  shift &             44  &             50  &             56  &     \bftab  74  \\
\hline
\end{tabular}
\caption{Modelled cycle counts for multiplication in $\F_{2^{128}}$ as used by \ALG{GHASH}.}
\label{tab:gcm:cycles}
\end{table}

\begin{table}[p]
\centering
\begin{tabular}{|c|l|rr|r|r|}
\hline
  \multicolumn{1}{|c|}{ISA}
& \multicolumn{1}{ c|}{Variant}
& \multicolumn{1}{ c|}{             ISE}
& \multicolumn{1}{ c|}{       ISE Path }
& \multicolumn{1}{ c|}{\CORE{2}     CPU}
& \multicolumn{1}{ c|}{\CORE{1}     CPU}
\\
& \multicolumn{1}{ c|}{/ Goal       }
& \multicolumn{1}{ c|}{Area         }
& \multicolumn{1}{ c|}{Depth        }
& \multicolumn{1}{ c|}{$+$ ISE area }
& \multicolumn{1}{ c|}{$+$ ISE area }
\\
\hline
\hline
 RV32IMC & Baseline    &              &            &       37325  ($1.00\times$) &       3501576 ($1.000\times$) \\
 RV32IMC & \ISE{1} (L) &        1605  & \bftab 17  &       39154  ($1.05\times$) &       3506224 ($1.001\times$) \\
 RV32IMC & \ISE{1} (A) &        1038  &        23  &       38561  ($1.05\times$) &       3505695 ($1.001\times$) \\
 RV32IMC & \ISE{2} (L) &        1611  & \bftab 17  &       40337  ($1.03\times$) &       3506729 ($1.001\times$) \\
 RV32IMC & \ISE{2} (A) &         780  &        21  &       38479  ($1.08\times$) &       3505910 ($1.001\times$) \\
 RV32IMC & \ISE{3}     & \bftab  630  &        25  &\bftab 38301  ($1.03\times$) &       3506097 ($1.001\times$) \\
 RV32IMC & \ISE{5} (L) &        1852  &        23  &       40626  ($1.03\times$) &       3507518 ($1.001\times$) \\
 RV32IMC & \ISE{5} (A) &        1048  &        23  &       38749  ($1.09\times$) &       3506816 ($1.001\times$) \\
\hline
\hline
 RV64IMC & Baseline &          &          &  N/A  & 3717607 (1.000$\times$) \\
 RV64IMC & \ISE{4}  &  3790    &    27    &  N/A  & 3728235 (1.003$\times$) \\
\hline
\end{tabular}
\caption{Hardware implementation metrics for each ISE variant with only encrypt instructions implemented.}
\label{tab:eval:hw:dec}
\end{table}

% -----------------------------------------------------------------------------

\noindent
The Galois/Counter Mode (GCM) ~\cite{NIST:sp.800.38d}
is a block cipher mode of operation which 
supports authenticated encryption.
AES-GCM refers to an instantiation using AES as the underlying block cipher, 
which is the only case mandated by TLS 1.3~\cite[Section 9.1]{rfc:8446}; the
importance of this construction means GCM and AES are frequently considered 
together from an implementation and evaluation perspective.
The computational core of AES-GCM is formed from two components.
\ALG{GCTR} ~\cite[Section 6.5]{NIST:sp.800.38d}
is responsible for 
    encryption
using AES,
and
\ALG{GHASH}~\cite[Section 6.4]{NIST:sp.800.38d}
is responsible for
authentication.
Having dealt with efficient implementation of AES and hence \ALG{GCTR} in
\REFSEC{sec:ise}, we turn our attention to \ALG{GHASH}.  
Rather than further 
embellish the ISE for AES, we instead focus on re-use of the proposed
standard 
Bit-manipulation ~\cite[Section 17]{RV:ISA:I:19}
(At the time of writing, the draft extension proposal is found in
\cite{riscv:bitmanip:draft})
extension.
This approach is attractive for two reasons.
AES-GCM is a very common construction, but AES is not the only block
cipher which can be used with GCM.
Likewise, AES may not always be used with GCM, so separation of
the two constructs from an instruction set point of view is prudent.

% -----------------------------------------------------------------------------

\paragraph{Implementation.}

\ALG{GHASH}~\cite[Section 6.4]{NIST:sp.800.38d} is a universal hash defined 
over the finite field $\F_{2^{128}}$ constructed as
$
\F_{2}[\IND{x}] / ( \IND{x}^{128} + \IND{x}^{7} + \IND{x}^{2} + \IND{x} + 1 ) .
$
Conversion of the input into the correct endianness can be realised using
the 
\VERB{grev} (or generalised reverse)
instruction,
which can reverse the bits in each byte of an input word:
$4$ (resp. $2$) 
\VERB{grev} 
instructions are therefore required on RV32IB (resp. RV64IB).
Beyond this, operations in $\F_{2^{128}}$ dominate.
Addition       in $\F_{2^{128}}$ 
is equivalent to XOR: thus
$4$ (resp. $2$) 
\VERB{xor} 
instructions are required on RV32IB (resp. RV64IB).
Multiplication in $\F_{2^{128}}$ 
can be split into two steps:
a $( 128 \times 128 )$-bit polynomial multiplication, 
followed by 
a reduction of the $256$-bit result modulo
$
\IND{x}^{128} + \IND{x}^{7} + \IND{x}^{2} + \IND{x} + 1 .
$

The multiplication step 
can be realised using pairs of ``carry-less'' multiplication instructions
\VERB{clmul} and \VERB{clmulh}.
These compute the least significant (resp. most-significant) 
half of a carry-less product (i.e., product over $\F_2$).
Pairs of 
\VERB{clmul} and \VERB{clmulh}
should be scheduled adjacently, allowing capable micro-architectures
to fuse them.
Use of a school book approach 
requires
$16$ (resp. $4$) pairs 
on RV32IB (resp. RV64IB).
Optimisation using the Karatsuba method
requires
$ 9$ (resp. $3$) such pairs 
on RV32IB (resp. RV64IB),
plus some additional \VERB{xor} instructions.

The reduction step
can be implemented in two ways:
a shift-based reduction, made possible by the low Hamming weight of the
primitive polynomial,
or
a multiplication-based reduction, analogous to the Montgomery or Barret
methods.
The most efficient approach depends on the relative execution 
latency of
\VERB{clmul[h]}
vs.
\VERB{xor} and \VERB{s[lr]li}.
Note that the entire \ALG{GHASH} operation, including \VERB{clmul[h]},
{\em must} exhibit data-oblivious execution latency 
(e.g., avoid data-dependent optimisations like early-termination)
to avoid associated side-channel attacks (cf.~\cite{GOPT:09}).

% -----------------------------------------------------------------------------

\paragraph{Discussion.}

\REFTAB{tab:gcm:instrs} 
lists instruction counts for 
multiplication in $\F_{2^{128}}$,
implemented using combinations of the base ISA, and approaches
for the polynomial multiplication and reduction steps.
\REFTAB{tab:gcm:cycles} 
then models the execution latency 
(measured in cycles)
assuming \VERB{grev}, \VERB{xor}, and \VERB{s[lr]li} take $1$ cycle.
Although the model only considers an in-order core in line with those used
in \REFSEC{sec:ise} and is focused on execution latency
(vs. other pertinent metrics, such as code footprint),
there are two obvious conclusions:
if
\VERB{clmul[h]}
has $2$ (or more) times the latency of
\VERB{xor} and \VERB{s[lr]li},
a 
Karatsuba
polynomial multiplication
is preferable.
If
\VERB{clmul[h]}
has $6$ (or more) times the latency of
\VERB{xor} and \VERB{s[lr]li},
a shift-based 
reduction 
is preferable.

We recommend the carry-less multiply instructions
specified in the proposed RISC-V Bit-manipulation extension also be included
in the RISC-V cryptography extension.
Implementers would otherwise need to implement (a subset of) the B
extension, potentially adding functionality and cost that is not
necessary.

An important consideration for the GCTR component of GCM is that it only
requires the encryption function for a block cipher.
Given this, we re-evaluate the hardware costs of each ISE, assuming that
only the encryption instructions are implemented.
These results are shown in \REFTAB{tab:eval:hw:dec}.
Unsurprisingly, the area overhead is approximately halved, and there is
a small reduction in circuit depth.
For very constrained devices which have exact functionality
requirements, we believe that making implementation of the decryption
instruction optional could be beneficial.
If these systems {\em do} require AES decryption, it could still be
implemented in software, with a performance and code size similar
to the baseline implementations in
\REFTAB{tab:eval:sw:perf:2}
and
\REFTAB{tab:eval:sw:perf:1}.

% =============================================================================


% =============================================================================

% NOTE: Commented out as SCA section not likely to be included in final version.
%\section{Hardening an AES ISE against DPA attack}
%\label{sec:sca}
%\input{tex/body-sca.tex}

% =============================================================================

\section{Conclusion}
\label{sec:outro}
% =============================================================================

%Although differing in nature, both AES and RISC-V represent important
%standards.  In this paper, we have addressed the challenge of secure, 
%efficient implementation of AES on RISC-V: our approach harnesses the
%modularity afforded by RISC-V, through a focus on the use of ISEs.

Motivated by ongoing efforts to standardise support 
for AES in RISC-V, we have implemented and evaluated five ISE designs 
on two different RISC-V compliant base micro-architectures.
Our conclusion is that
1) \ISE{3}
   is the best option for 
   AES on $32$-bit cores,
2) \ISE{4}
   is the best option for 
   AES on $64$-bit cores,
   and
3) the
   standard 
   B~\cite[Section 17]{RV:ISA:I:19}
   extension
   can combine with either option to support AES-GCM.

\paragraph{Future Work}
Our evaluations of the different ISEs have foucsed primarily on
performance, code size and hardware cost metrics.
Because our work is a departure from historic
AES ISEs in that they are designed to be suitable for small, embedded
CPUs, power and EM side-channel security will likely be a consideration
for implementations of these ISEs.
We consider the problem of side-channel secure ISE design to be an
open problem, particularly in terms of making the same code portably
side-channel secure across multiple implementations of the same ISE.
Future efforts would be well spent in studying this problem, perhaps
looking at creating custom extensions based on the recommendations here
to support side-channel security.

% TODO: maybe try to add some insight into ISA design constraints ...
%
%The requirement for $3$-address (i.e., $2$ source and $1$ destination)
%instruction format prevented some further optimisations, 
%e.g., the integration of \AESFUNC{AddRoundKey} in \ISE{4}.

% =============================================================================


% ============================================================================

\ifbool{anonymous}{}{%
\section*{Acknowledgements}

We would like to thank the reviewers for their helpful and 
constructive comments.

This work was undertaken as part of the ongoing standardisation of
RISC-V. We are grateful to all members of
the Cryptographic Extensions Task Group
who contributed to related discussions, particularly Andy Glew and
Barry Spinney.
The opinions expressed in this paper are the author's alone, not
of their respective employers or RISC-V International.
The RISC-V cryptography extension is in the process of being standardised
at the time of writing. The purpose of this work is to support that process.

The first and third authors were supported in part by
EPSRC via grant EP/R012288/1, under the
RISE (\url{http://www.ukrise.org}) programme
and
Innovate UK Project 105747.
}%

% =============================================================================

\bibliographystyle{alpha}
\bibliography{paper}

% =============================================================================

\appendix

\clearpage
\section{Example AES round function implementations}
\label{sec:round-functions}
\begin{figure}[!h]
\begin{lstlisting}[language=pseudo,style=block]
lw           a0,  0(a4)       // Load Round Key
lw           a1,  4(a4)
lw           a2,  8(a4)
lw           a3, 12(a4)
xor          a4, a4, a0       // Add Round Key
xor          a5, a5, a1
xor          a6, a6, a2
xor          a7, a7, a3
saes.v1.encs a0, a4           // SubBytes
saes.v1.encs a1, a5
saes.v1.encs a2, a6
saes.v1.encs a3, a7
                              // Shift Rows
and          a4, t0, t6   ; and   a5, t1, t6
and          a6, t2, t6   ; and   a7, t3, t6
slli         t4, t6, 0x8  ; and   t5, t0, t4
or           a7, a7, t5   ; and   t5, t3, t4
or           a6, a6, t5   ; and   t5, t2, t4
or           a5, a5, t5   ; and   t5, t1, t4
or           a4, a4, t5   ; slli  t4, t4, 0x8
and          t5, t2, t4   ; or    a4, a4, t5
and          t5, t3, t4   ; or    a5, a5, t5
and          t5, t0, t4   ; or    a6, a6, t5
and          t5, t1, t4   ; or    a7, a7, t5
slli         t4, t4, 0x8  ; and   t5, t3, t4
or           a4, a4, t5   ; and   t5, t0, t4
or           a5, a5, t5   ; and   t5, t1, t4
or           a6, a6, t5   ; and   t5, t2, t4
or           a7, a7, t5
saes.v1.encm t0, a4           // MixColumns
saes.v1.encm t1, a5
saes.v1.encm t2, a6
saes.v1.encm t3, a7
\end{lstlisting}
\caption{
  An AES encryption round implemented using \ISE{1}.
}
\label{fig:v1:round}
\end{figure}

\begin{figure}[!h]
\begin{lstlisting}[language=pseudo,style=block]
lw              a0,  0(a4)     // Load Round Key
lw              a1,  4(a4)
lw              a2,  8(a4)
lw              a3, 12(a4)
xor             t0, t0, a0     // Add Round Key
xor             t1, t1, a1
xor             t2, t2, a2
xor             t3, t3, a3
saes.v2.sub.enc a0, t0, t1     // SubBytes / ShiftRows
saes.v2.sub.enc a1, t2, t3
saes.v2.sub.enc a2, t1, t2
saes.v2.sub.enc a3, t3, t0
saes.v2.mix.enc t0, a0, a1     // ShiftRows / MixColumns
saes.v2.mix.enc t1, a2, a3
saes.v2.mix.enc t2, a1, a0
saes.v2.mix.enc t3, a3, a2
\end{lstlisting}
\caption{
  An AES encryption round implemented using \ISE{2}.
}
\label{fig:v2:round}
\end{figure}

\begin{figure}[!h]
\begin{lstlisting}[language=pseudo,style=block]
lw              a0, 16(RK)      // Load Round Key
lw              a1, 20(RK)
lw              a2, 24(RK)
lw              a3, 28(RK)      // t0,t1,t2,t3 contains current round state.
saes.v3.encsm   a0, a0, t0, 0   // Next state for column 0.
saes.v3.encsm   a0, a0, t1, 1   // Current column 0 in t0.
saes.v3.encsm   a0, a0, t2, 2   // Next column 0 accumulates in a0
saes.v3.encsm   a0, a0, t3, 3
saes.v3.encsm   a1, a1, t1, 0   // Next state for column 1.
saes.v3.encsm   a1, a1, t2, 1
saes.v3.encsm   a1, a1, t3, 2
saes.v3.encsm   a1, a1, t0, 3
saes.v3.encsm   a2, a2, t2, 0   // Next state for column 2.
saes.v3.encsm   a2, a2, t3, 1
saes.v3.encsm   a2, a2, t0, 2
saes.v3.encsm   a2, a2, t1, 3
saes.v3.encsm   a3, a3, t3, 0   // Next state for column 3.
saes.v3.encsm   a3, a3, t0, 1
saes.v3.encsm   a3, a3, t1, 2
saes.v3.encsm   a3, a3, t2, 3   // a0,a1,a2,a3 contains new round state
\end{lstlisting}
\caption{
  An AES encryption round implemented using \ISE{3}.
}

\label{fig:v3:round}
\end{figure}
\begin{figure}[!h]
\begin{lstlisting}[language=pseudo,style=block]
ld              a0, 0(a4)  // Load round key as double words.
ld              a1, 8(a4)
xor             t0, t0, a0 // Add round key for 2 columns at a time.
xor             t1, t1, a1
aes.v4.encsm    t2, t0, t1 // Next round state: columns 0, 1
aes.v4.encsm    t3, t1, t0 // columns 2, 3 - Note swapped rs1/rs2
\end{lstlisting}
\caption{
  An AES encryption round implemented using \ISE{4}.
}
\label{fig:v4:round}
\end{figure}

\begin{figure}[!h]
\begin{lstlisting}[language=pseudo,style=block]
lw                a0,  0(a4)   // Load Round Key
lw                a1,  4(a4)
lw                a2,  8(a4)
lw                a3, 12(a4)
xor               t0, t0, a0   // Add Round Key
xor               t1, t1, a1
xor               t2, t2, a2
xor               t3, t3, a3
saes.v5.esrsub.lo a0, t0, t1   // Quad 0: SubBytes / ShiftRows
saes.v5.esrsub.lo a1, t1, t0   // Quad 1
saes.v5.esrsub.hi a2, t2, t3   // Quad 2
saes.v5.esrsub.hi a3, t3, t2   // Quad 3
saes.v5.emix      t0, a0, a2   // Quad 0: ShiftRows / MixColumns
saes.v5.emix      t1, a1, a3   // Quad 1
saes.v5.emix      t2, a2, a0   // Quad 2
saes.v5.emix      t3, a3, a1   // Quad 3
\end{lstlisting}
\caption{
  An AES encryption round implemented using \ISE{5}.
}
\label{fig:v5:round}
\end{figure}

\clearpage
\section{Additional technical detail for CPU cores}
\label{sec:cores}
% =============================================================================

\begin{figure}[!h]
\centering
\includegraphics[scale={0.45},angle={90}]{diagrams/scarv-cpu-uarch.png}
\caption{
  \CORE{2} core micro-architecture.
}
\label{fig:core:2:normal}
\end{figure}

\begin{figure}[!h]
\begin{lstlisting}[style={block},language={scala}]
class AESVanilla32 extends Config (
  new freechips.rocketchip.subsystem.WithNoMMIOPort ++
  new freechips.rocketchip.subsystem.WithNoSlavePort ++
  new freechips.rocketchip.subsystem.WithInclusiveCache ++
  new freechips.rocketchip.subsystem.WithRV32 ++
  new freechips.rocketchip.subsystem.WithNExtTopInterrupts(0) ++
  new freechips.rocketchip.subsystem.WithNBigCores(1) ++
  new freechips.rocketchip.subsystem.WithoutFPU ++
  new freechips.rocketchip.system.BaseConfig
)
\end{lstlisting}
\caption{$32$-bit \CORE{1} core configuration.}
\label{fig:rocket:32}
\end{figure}

\begin{figure}[!h]
\begin{lstlisting}[style={block},language={scala}]
class AESVanilla64 extends Config(
  new freechips.rocketchip.subsystem.WithNoMMIOPort ++
  new freechips.rocketchip.subsystem.WithNoSlavePort ++
  new freechips.rocketchip.subsystem.WithInclusiveCache ++
  new freechips.rocketchip.subsystem.WithNExtTopInterrupts(0) ++
  new freechips.rocketchip.subsystem.WithNBigCores(1) ++
  new freechips.rocketchip.subsystem.WithoutFPU ++
  new freechips.rocketchip.system.BaseConfig
)
\end{lstlisting}
\caption{$64$-bit \CORE{1} core configuration.}
\label{fig:rocket:64}
\end{figure}

% NOTE: Commented out as SCA section not likely to be included in final version.
%\begin{figure}[!h]
%\centering
%\includegraphics[scale=0.45,angle=90]{diagrams/scarv-cpu-uarch-sca.png}
%\caption{
%  \CORE{2} core: hardened micro-architecture, 
%  extending \ISE{3} for improved security against side-channel attack.
%  Connections coloured red are security-critical, in the sense they relate to masks.
%}
%\label{fig:core:2:secure}
%\end{figure}

% =============================================================================


%\clearpage
%\section{Additional algorithms}
%\label{sec:alg}
%\input{tex/appx-alg.tex}

% =============================================================================

\end{document}
